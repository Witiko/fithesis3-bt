% \iffalse meta-comment
% fithesis.dtx
% Copyright 1998--2015 Daniel Marek (DM), Jan Pavlovič (JP),
% Vít Novotný (VN), Petr Sojka (PS)
% http://www.fi.muni.cz/tech/unix/tex/fithesis.xhtml
% Faculty of Informatics, Masaryk University
%
% This work may be distributed and/or modified under the
% conditions of the LaTeX Project Public License, either version 1.3
% of this license or (at your option) any later version.
% The latest version of this license is in
%   http://www.latex-project.org/lppl.txt
% and version 1.3 or later is part of all distributions of LaTeX
% version 2005/12/01 or later.
%
% This work has the LPPL maintenance status `maintained'.
% 
% The Current Maintainer of this work is Vít Novotný.
% Send bug reports, requests for additions and questions
% to the fithesis discussion forum at
% <http://is.muni.cz/auth/df/fithesis-sazba/>.
%
% This work consists of the files fithesis.dtx and fithesis.ins
% and the derived files fithesis3.cls, fithesis2.cls, fithesis.cls,
% fit10.clo, fit11.clo, fit12.clo.
%
%    \begin{macrocode}
%<*driver>

\documentclass{ltxdoc}\setcounter{page}{175}
\usepackage[utf8]{inputenc} % this file uses UTF-8
\usepackage[english]{babel}
\usepackage{tgpagella}
\usepackage{tabularx}
\usepackage{hologo}
\usepackage{booktabs}
\usepackage[scaled=0.86]{berasans}
\usepackage[scaled=1.03]{inconsolata}
\usepackage[resetfonts]{cmap}
\usepackage[T1]{fontenc} % use 8bit fonts
\emergencystretch 2dd
\usepackage{hypdoc}

% Making paragraphs numbered
\makeatletter
\renewcommand\paragraph{\@startsection{paragraph}{4}{\z@}%
            {-2.5ex\@plus -1ex \@minus -.25ex}%
            {1.25ex \@plus .25ex}%
            {\normalfont\normalsize\bfseries}}
\makeatother
\setcounter{secnumdepth}{4} % how many sectioning levels to assign
\setcounter{tocdepth}{4}    % how many sectioning levels to show

% ltxdoc class options
\CodelineIndex
\MakeShortVerb{|}
\EnableCrossrefs
\DoNotIndex{}
\makeatletter
\c@IndexColumns=2
\makeatother

\begin{document}
  \RecordChanges
  \DocInput{fithesis.dtx}
  \PrintIndex
  \PrintChanges
\end{document}

%</driver>
%    \end{macrocode}
%<*class>
\NeedsTeXFormat{LaTeX2e}
% \fi
\def\thesis@version{2015/06/21 v0.3.16 fithesis3 MU thesis class}
%
%%%%%%%%%%%%%%%%%%%%%%%%%%%%%%%%%%%%%%%%%%%%%%%%%%%%%%%%%%%%%%%%%%%%%%%%%%%%%%%
%
% \changes{v0.3.16}  {2015/06/21}{Clubs and widows are now set to
%   be infinitely bad. The \texttt{assignment} key has weaker, but
%   more robust semantics now.}
% \changes{v0.3.15}  {2015/06/14}{Renamed \cs{thesis@requireStyle}
%   to \cs{thesis@requireWithOptions} and moved the style loader
%   from the \cs{thesis@load} routine to a new
%   \cs{thesis@requireStyle} macro to make the semantics of
%   \cs{thesis@requireLocale} and \cs{thesis@requireStyle} more
%   similar. Changed the \texttt{basepath}, \texttt{logopath},
%   \texttt{localepath} and \texttt{stylepath} keys to match the
%   lower camelcasing of the rest of the keys. Added further
%   description regarding the use of the \texttt{assignment} key.
%   [VN]}
% \changes{v0.3.14}  {2015/06/07}{Updated the documentation. [VN]}
% \changes{v0.3.13}  {2015/05/30}{Fixed an inconsistency in the
%   example code. Removed an extraneous \cs{thesis@blocks@clear}
%   block withing the definition of \cs{thesis@blocks@frontMatter}
%   in the fss style file. Added comments, fixed clubs and widows
%   and removed text overflows within the user guides. Adjusted the
%   colors of various style files. Removed the trailing dot in the
%   bibliographic identification within the med and ped style
%   files. Fixed a typo within the technical documentation. Fixed
%   the twoside alignment of the \cs{thesis@blocks@bibEntry} and
%   the \cs{thesis@blocks@bibEntryEn} blocks within the sci style
%   file.  The \cs{thesis@blocks@assignment} block no longer clears
%   a page when nothing is inserted. It is also no longer
%   hard-coded to be hidden for rigorous theses. Instead, the
%   \cs{ifthesis@blocks@assignment} conditional can be set either
%   by the subsequently loaded style files or by the user. So far,
%   only the fi and sci style files set the conditional. [VN]}
% \changes{v0.3.12}  {2015/05/24}{The subsections and
%   subsubsections now use the correct \texttt{tocdepth}. [VN]}
% \changes{v0.3.11}  {2015/05/15}{Added hyphenation into the
%   technical documentation. Fixed an unterminated group. Polished
%   the text of the guide. Added the \texttt{palatino} and
%   \texttt{nopalatino} options. Stylistic changes to the text of
%   the technical documentation. \cs{thesis@subdir} is now robust
%   against relative paths. Accounted for French spacing in the
%   guide. Fixed the \texttt{thesis@english@facultyName} string.
%   Documentation refinements. [VN]}
% \changes{v0.3.10}  {2015/05/09}{Fixed a typo in the technical
%   documentation. Updated the \emph{Advanced usage} chapter of the
%   user guide. The required packaged listed in Section 2.2 of the
%   user guide are now always correct. Adjusted the footer spacing
%   in the styles of econ and fi. Added \emph{Advanced usage}
%   chapter to the user guide. Added the description of basic
%   options into the user guide. Added the \texttt{table} and
%   \texttt{oldtable} options. Added the \texttt{type} field to the
%   guide for completeness. [VN]}
% \changes{v0.3.09}  {2015/04/26}{A complete refactoring of the class. The class
%   was decomposed into a base class, locale files and style files. [VN]}
% \changes{v0.3.08}  {2015/03/04}{Fixed a non-terminated \cs{if} condition.
%   [VN] (backport of v0.2.18)\\Fixed mostly documentation errors reported
%   at the new fithesis discussion forum (-ti, eco$\rightarrow$econ, implicit
%   twocolumn, example extended (font setup), etc.). [PS] (backport of v0.2.17)}
% \changes{v0.3.07}  {2015/02/03}{Replaced the \cs{thesiswoman} command with
%   \cs{thesisgender}. [VN]}
% \changes{v0.3.06}  {2015/01/26}{Added the colorx package and the base colors
%   for each faculty. If the color option is specified, the tabular environment
%   gets redefined and uses the faculty colors to color alternating table rows
%   to improve readability. The hyperref links in the e-version are now likewise
%   colored according to the chosen faculty, in this case regardless of the
%   presence of the color option. Dropped the support for typesetting theses
%   outside MU. [VN]}
% \changes{v0.3.05}  {2015/01/21}{Added support for change typesetting.
%   Restructured the code to make it more amenable to literal programming.
%   Added support for \cs{CodelineIndex} typesetting. Added information about
%   the usage of \textsf{fithesis1} and \textsf{fithesis2} on the FI unix
%   machines. (backport of v0.2.16) [VN]\\Minor changes throughout the text,
%   added a link to the the fithesis forums [PS] (backport of v0.2.15@r14:15)}
% \changes{v0.3.04}  {2015/01/14}{Import the url package to allow for the use of
%   \cs{url} within the documentation. (backport of v0.2.15@r13) [VN]}
% \changes{v0.3.03}  {2015/01/14}{Small fixes (added \cs{relax} at
%   \cs{MainMatter}), generating both fithesis.cls (obsolete, loading
%   \texttt{fithesis2.cls}) and \texttt{fithesis2.cls}, minor doc edits,
%   version numbering of \texttt{.clo} fixed, switch to utf8 and ensuring that
%   \texttt{.dtx} compiles. Documentation adjusted to the status quo, added
%   link to discussion forum (backport of v0.2.14) [PS]}
% \changes{v0.3.02}  {2015/01/13}{pdf metadata stamping added for
%   \cs{thesistitle} and \cs{thesisstudent} [VN]}
% \changes{v0.3.01}  {2015/01/09}{documentation now uses babel and cmap
%   packages. the entire file was transcoded into utf8, \cs{thesiscolor} was
%   replaced by color class option, added pdf metadata stamping support [VN]}
% \changes{v0.3.00}  {2015/01/01}{fi logo is no longer special-cased (added eps
%   and pdf), \cs{thesislogopath} added to set the logo directory path,
%   \cs{thesiscolor} added to enable colorful typo elements [VN]}
% \changes{v0.2.12a}{2008--2011}{fork fithesis2 by Mr. Filipčík and Janoušek;
%   cf. \protect\url{http://github.com/liskin/fithesis}}
% \changes{v0.2.12} {2008/07/27}{Licence change to the LPPL [JP]}
% \changes{v0.2.11} {2008/01/07}{fix missing \texttt{fi-logo.mf} [JP,PS]}
% \changes{v0.2.10} {2006/05/12}{fix EN name of Acknowledgement [JP]}
% \changes{v0.2.09}  {2006/05/08}{add EN version of University name [JP]}
% \changes{v0.2.08}  {2006/01/20}{add change of University name [JP]}
% \changes{v0.2.07}  {2005/05/10}{escape all Czech letters [JP]
%   babel is used instead of stupid package czech [JP]
%   \cs{MainMatter} should be placed after \cs{tablesofcontents} [PS]}
% \changes{v0.2.06}  {2004/12/22}{fix : behind Advisor [JP]}
% \changes{v0.2.05}  {2004/05/13}{add English abstract [JP]}
% \changes{v0.2.04}  {2004/05/13}{fix SK declaration [Peter Cerensky, JP]}
% \changes{v0.2.03}  {2004/05/13}{fix title spacing [PS, JP]}
% \changes{v0.2.02}  {2004/05/12}{fix encoding bug [JP]}
% \changes{v0.2.01}  {2004/05/11}{add subsubsection to toc [JP]}
% \changes{v0.2.00}  {2004/05/03}{add sk lang [JP, Peter Cerensky]
%   set default cls class to \textsf{rapport3} [JP]}
% \changes{v0.1g}   {2004/04/01}{change of default size (12pt$\rightarrow$11pt) [JP]}
% \changes{v0.1f}   {2004/01/24}{add documentation for hyperref [JP]}
% \changes{v0.1e}   {2004/01/07}{add Brno to MU title [JP]}
% \changes{v0.1d}   {2003/03/24}{removed def schapter from fit1*.clo [JP]}
% \changes{v0.1c}   {2003/02/21}{default values of \cs{facultyname} and
%   \\\cs{@thesissubtitle} set for backward compatibility) [PS]}
% \changes{v0.1b}   {2003/02/14}{change of default size (11pt$\rightarrow$12pt) [JP]}
% \changes{v0.1a}   {2003/02/12}{minor documentation changes (CZ only,
%   sorry) [PS]}
% \changes{v0.1}    {2003/02/11}{new release, documentation editing (CZ only,
%   sorry) [PS]}
% \changes{v0.0a}   {2002}{changes by Jan Pavlovič to allow fithesis being
%   backend of docbook based system for thesis writing}
% \changes{v0.0}    {1998}{bachelor project of Daniel Marek under
%   supervision of Petr Sojka}
%
%%%%%%%%%%%%%%%%%%%%%%%%%%%%%%%%%%%%%%%%%%%%%%%%%%%%%%%%%%%%%%%%%%%%%%%%%%%%%%%
%
% \title{The \textsf{fithesis3} class for the typesetting of theses written
%   at the Masaryk University in Brno}
% \author{Daniel Marek, Jan Pavlovič, Vít Novotný, Petr Sojka}
% \date{\today}
% \maketitle
%
% \begin{abstract}
% \noindent This document details the design and the implementation
% of the \textsf{fithesis3} document class. It contains technical
% information for anyone who wishes to extend the class with their
% locale or style files. Users who only wish to use the class are
% advised to consult the guides distributed along with the class,
% which only document the parts of the public API relevant to the
% given style files.
% \end{abstract}
%
% \tableofcontents
%
% \section{Required classes and packages}
% The class loads the \texttt{rapport3} base class and the
% following packages: \begin{itemize}
%   \item\textsf{keyval} -- Adds support for parsing
%     comma-delimited lists of key-value pairs.
%   \item\textsf{etoolbox} -- Adds support for expanding
%     code after the preamble using the |\AtPreamble| hook.
%   \item\textsf{ifxetex} -- Used to detect the \Hologo{XeTeX}
%     engine.
%   \item\textsf{ifluatex} -- Used to detect the \Hologo{LuaTeX}
%     engine.
%   \item\textsf{inputenc} -- Used to enable the input UTF-8
%     encoding. This package does not get loaded under
%     the \Hologo{XeTeX} and \Hologo{LuaTeX} engines.
% \end{itemize}
% The \texttt{hyperref} package is also conditionally loaded during
% the expansion of the |\thesis@load| macro (see Section
% \ref{sec:thesis@load}). Other packages may be required by the
% style files (see Section \ref{sec:style-files}) you are using.
%    \begin{macrocode}
\ProvidesClass{fithesis3}[\thesis@version]
\LoadClass[a4paper]{rapport3}
\RequirePackage{keyval}
\RequirePackage{etoolbox}
\RequirePackage{ifxetex}
\RequirePackage{ifluatex}
\ifxetex\else\ifluatex\else
  \RequirePackage[utf8]{inputenc}
\fi\fi
%    \end{macrocode}
% \section{Public API}
% \label{sec:public-api}
% \subsection{Options}
% Any \oarg{options} passed to the class will be handed down to the
% loaded style files. The supported options are therefore documented
% in the subsections of Section \ref{sec:style-files} dedicated to
% the respective style files.
%
% \subsection{The \cs{thesissetup} macro}
% \begin{macro}{\thesissetup}
% The main public macro is the |\thesissetup|\marg{keyvals}
% command, where \textit{keyvals} is a comma-delimited list of
% key-value pairs as defined by the \textsf{keyval} package. This
% macro needs to be included prior to the beginning of a \LaTeX\ 
% document. When used, the \textit{keyvals} are processed.
%
% Note that the values passed to the |\thesissetup| public macro
% may only contain one paragraph of text. If you wish to set
% multiple paragraphs of text as the value, you need to use
% the |\thesislong| public macro (see Section
% \ref{sec:thesislong}).
%    \begin{macrocode}
\def\thesissetup#1{%
  \setkeys{thesis}{#1}}
%    \end{macrocode}
% \subsubsection{The \texttt{basePath} key}
% \begin{macro}{\thesis@basepath}
% The \marg{\texttt{basePath}=path} pair sets the \textit{path}
% containing the class files. The \textit{path} is prepended to
% every other path (|\thesis@logopath|, |\thesis@stylepath| and
% |\thesis@localepath|) used by the class. If non-empty, the
% \textit{path} gets normalized to \textit{path/}. The normalized
% \textit{path} is stored within the private |\thesis@basepath| macro,
% whose implicit value is |fithesis3/|.
%    \begin{macrocode}
\def\thesis@basepath{fithesis3/}
\define@key{thesis}{basePath}{%
  \ifx\thesis@empty#1\thesis@empty%
    \def\thesis@basepath{}%
  \else%
    \def\thesis@basepath{#1/}%
  \fi}
%    \end{macrocode}
% \end{macro}
% \begin{macro}{\thesis@logopath}
% \subsubsection{The \texttt{logoPath} key}
% The \marg{\texttt{logoPath}=path} pair sets the \textit{path}
% containing the logo files, which is used by the style files
% loading the logo. The \textit{path} is normalized using the
% private |\thesis@subdir| macro and stored within the private
% |\thesis@logopath| macro, whose implicit value is
% |\thesis@basepath| followed by |logo/\thesis@university/|. By
% default, this expands to \texttt{fithesis3/logo/mu/}.
%    \begin{macrocode}
\def\thesis@logopath{\thesis@basepath logo/\thesis@university/}
\define@key{thesis}{logoPath}{%
  \def\thesis@logopath{\thesis@subdir#1%
    \empty\empty\empty\empty}}
%    \end{macrocode}
% \end{macro}
% \begin{macro}{\thesis@stylepath}
% \subsubsection{The \texttt{stylePath} key}
% The \marg{\texttt{stylePath}=path} pair sets the \textit{path}
% containing the style files. The \textit{path} is normalized using
% the private |\thesis@subdir| macro and stored within the private
% |\thesis@stylepath| macro, whose implicit value is
% |\thesis@basepath style/|. By default, this expands to
% \texttt{fithesis3/style/}.
%    \begin{macrocode}
\def\thesis@stylepath{\thesis@basepath style/}
\define@key{thesis}{stylePath}{%
  \def\thesis@stylepath{\thesis@subdir#1%
    \empty\empty\empty\empty}}
%    \end{macrocode}
% \end{macro}
% \begin{macro}{\thesis@localepath}
% \subsubsection{The \texttt{localePath} key}
% The \marg{\texttt{localePath}=path} pair sets the \textit{path}
% containing the locale files. The \textit{path} is normalized
% using the private |\thesis@subdir| macro and stored within the
% private |\thesis@localepath| macro, whose implicit value is
% |\thesis@basepath| followed by |locale/|.  By default, this
% expands to \texttt{fithesis3/locale/}.
%    \begin{macrocode}
\def\thesis@localepath{\thesis@basepath locale/}
\define@key{thesis}{localePath}{%
  \def\thesis@localepath{\thesis@subdir#1%
    \empty\empty\empty\empty}}
%    \end{macrocode}
% \end{macro}
% \begin{macro}{\thesis@subdir}
% The |\thesis@subdir| private macro returns |/| unchanged, coerces
% |.|, |..|, |/|\textit{path}, |./|\textit{path} and
% |../|\textit{path} to |./|, |../|, |/|\textit{path}|/|,
% |./|\textit{path}|/| and |../|\textit{path}|/|, respectively, and
% coerces any other \textit{path} into |\thesis@basepath|
% \textit{path}|/|.
%    \begin{macrocode}
\def\thesis@subdir#1#2#3#4\empty{%
  \ifx#1\empty%           <empty> -> <basepath>
    \thesis@basepath
  \else
    \if#1/%
      \ifx#2\empty%             / -> /
        /%
      \else%              /<path> -> /<path>/
        #1#2#3#4/%
      \fi
    \else%
      \if#1.%
        \ifx#2\empty%           . -> ./
          ./%
        \else
          \if#2.%
            \ifx#3\empty%      .. -> ../
              ../%
            \else
              \if#3/%   ../<path> -> ../<path>/
                ../#4/%
              \else
                \thesis@basepath#1#2#3#4/%
              \fi
            \fi
          \else
            \if#2/%      ./<path> -> ./<path>/
              ./#3#4/%
            \else
              \thesis@basepath#1#2#3#4/%
            \fi
          \fi
        \fi
      \else
        \thesis@basepath#1#2#3#4/%
      \fi
    \fi%
  \fi}
%    \end{macrocode}
% \end{macro}
% \begin{macro}{\thesis@def}
% The |\thesis@def|\oarg{key}\marg{name} private macro defines
% the private |\thesis@|\textit{name} macro to expand
% to either <<\textit{key}>>, if specified, or to
% <<\textit{name}>>. The macro serves to provide the placeholder
% string for user-defined macros with no default value.
%    \begin{macrocode}
\newcommand{\thesis@def}[2][]{%
  \expandafter\def\csname thesis@#2\endcsname{%
    <<\ifx\thesis@empty#1\thesis@empty#2\else#1\fi>>}}
%    \end{macrocode}
% \end{macro}
% \begin{macro}{\thesis@declaration}
% \subsubsection{The \texttt{declaration} key}
% The \marg{\texttt{declaration}=text} pair sets the
% declaration \textit{text} to be included into the document.
% \cmd{/thesis@basepath} followed by \textit{path}. The
% \textit{text} is stored within the private |\thesis@declaration|
% macro, whose implicit value is |\thesis@@{declaration}|.
%    \begin{macrocode}
\def\thesis@declaration{\thesis@@{declaration}}
\long\def\KV@thesis@declaration#1{%
  \long\def\thesis@declaration{#1}}
%    \end{macrocode}
% \end{macro}
% \begin{macro}{\ifthesis@woman}
% \subsubsection{The \texttt{gender} key}
% The \marg{\texttt{gender}=char} pair sets the author's gender to
% either a male, if \textit{char} is the character \texttt{m}, or
% to a female. The gender can be tested using the
% |\ifthesis@woman| \ldots |\else| \ldots |\fi| conditional. The
% implicit gender is male.
%    \begin{macrocode}
\newif\ifthesis@woman\thesis@womanfalse
\define@key{thesis}{gender}{%
  \def\thesis@male{m}%
  \def\thesis@arg{#1}%
  \ifx\thesis@male\thesis@arg%
    \thesis@womanfalse%
  \else%
    \thesis@womantrue%
  \fi}
%    \end{macrocode}
% \end{macro}
% \begin{macro}{\thesis@author}
% \subsubsection{The \texttt{author} key}
% The \marg{\texttt{author}=name} pair sets the author's full
% name to \textit{name}. The \textit{name} is parsed using the
% \DescribeMacro{\thesis@parseAuthor} private macro and stored
% within the following private macros:
% \begin{itemize}
%   \item\DescribeMacro{\thesis@author}|\thesis@author|
%     -- The full name of the author.
%   \item\DescribeMacro{\thesis@author@head}|\thesis@author@head|
%     -- The first space-delimited part of the name. This
%     corresponds to the author's first name.
%   \item\DescribeMacro{\thesis@author@tail}|\thesis@author@tail|
%     -- The full name without the first space-delimited part of
%     the name. This corresponds to the author's surname.
% \end{itemize}
%    \begin{macrocode}
\def\thesis@parseAuthor#1{%
  \def\thesis@author{#1}%
  \def\thesis@author@head{\expandafter\expandafter\expandafter%
    \@gobble\thesis@head#1 \relax}%
  \def\thesis@author@tail{\thesis@tail#1 \relax}}
\thesis@def{author}%
\thesis@def[author]{author@head}%
\thesis@def[author]{author@tail}%
\define@key{thesis}{author}{%
  \thesis@parseAuthor{#1}}
%    \end{macrocode}
% \end{macro}
% \begin{macro}{\thesis@id}
% \subsubsection{The \texttt{id} key}
% The \marg{\texttt{id}=identifier} pair sets the identifier
% of the thesis author to \textit{identifier}. This usually
% corresponds to a unique identifier of the author within the
% information system of the given university.
%    \begin{macrocode}
\thesis@def{id}
\define@key{thesis}{id}{%
  \def\thesis@id{#1}}
%    \end{macrocode}
% \end{macro}
% \begin{macro}{\thesis@type}
% \subsubsection{The \texttt{type} key}
% The \marg{\texttt{type}=type} pair sets the type of the thesis
% to \textit{type}. The following types of theses are recognized:
% \begin{center}\begin{tabular}{lc}\toprule
%   The thesis type & The value of \textit{type} \\\midrule
%   Bachelor's thesis & \texttt{bc} \\
%   Master's thesis & \texttt{mgr} \\
%   Doctoral thesis & \texttt{d} \\
%   Rigorous thesis & \texttt{r} \\\bottomrule
% \end{tabular}\end{center}
% The \textit{type} is stored within the private |\thesis@type|
% macro, whose implicit value is |bc|. For the ease of testing of
% the thesis type via |\ifx| conditions within style and locale
% files, the \DescribeMacro{\thesis@bachelors}|\thesis@bachelors|,
% \DescribeMacro{\thesis@masters}|\thesis@masters|,
% \DescribeMacro{\thesis@doctoral}|\thesis@doctoral| and
% \DescribeMacro{\thesis@rigorous}|\thesis@rigorous| macros
% containing the corresponding \textit{type} values are available
% as a part of the private API.
%    \begin{macrocode}
\def\thesis@bachelors{bc}
\def\thesis@masters{mgr}
\def\thesis@doctoral{d}
\def\thesis@rigorous{r}
\let\thesis@type\thesis@bachelors
\define@key{thesis}{type}{%
  \def\thesis@type{#1}}
%    \end{macrocode}
% \end{macro}
% \begin{macro}{\thesis@university}
% \subsubsection{The \texttt{university} key}
% The \marg{\texttt{university}=id} pair sets the identifier of
% the university, at which the thesis is being written,
% to \textit{id}. The \textit{id} is stored within the private
% |\thesis@university| macro, whose implicit value is \texttt{mu}.
% The |\thesis@university|
% macro is used by the |\thesis@logopath| macro and when loading
% the style and locale files using the |\thesis@load| macro. It
% allows for the usage of the class at universities other than
% the Masaryk University in Brno without the need to alter the
% code.
%    \begin{macrocode}
\def\thesis@university{mu}
\define@key{thesis}{university}{%
  \def\thesis@university{#1}}
%    \end{macrocode}
% \end{macro}
% \begin{macro}{\thesis@faculty}
% \subsubsection{The \texttt{faculty} key}
% The \marg{\texttt{faculty}=domain} pair sets the faculty, at
% which the thesis is being written, to \textit{domain}. The
% following \textit{domain} names are recognized:
% \begin{center}\begin{tabularx}{\textwidth}{Xc}\toprule
%   The Faculty & The \textit{domain} name \\\midrule
%   The Faculty of Informatics & \texttt{fi} \\
%   The Faculty of Science & \texttt{sci} \\
%   The Faculty of Law & \texttt{law} \\
%   The Faculty of Economics and Administration & \texttt{econ} \\
%   The Faculty of Social Studies & \texttt{fss} \\
%   The Faculty of Medicine & \texttt{med} \\
%   The Faculty of Education & \texttt{ped} \\
%   The Faculty of Arts & \texttt{phil} \\
%   The Faculty of Sports Studies & \texttt{fsps} \\\bottomrule
% \end{tabularx}\end{center}
% The \textit{domain} name is stored within the private
% |\thesis@faculty| macro, whose implicit value is \texttt{fi}.
%    \begin{macrocode}
\def\thesis@faculty{fi}
\define@key{thesis}{faculty}{%
  \def\thesis@faculty{#1}}
%    \end{macrocode}
% \end{macro}
% \begin{macro}{\thesis@department}
% \subsubsection{The \texttt{department} key}
% The \marg{\texttt{department}=name} pair sets the name of the
% department, at which the thesis is being written, to
% \textit{name}. The \textit{name} is stored within the private
% |\thesis@department| macro.
%    \begin{macrocode}
\thesis@def{department}
\define@key{thesis}{department}{%
  \def\thesis@department{#1}}
%    \end{macrocode}
% \end{macro}
% \begin{macro}{\thesis@departmentEn}
% \subsubsection{The \texttt{departmentEn} key}
% The \marg{\texttt{departmentEn}=name} pair sets the English
% name of the department, at which the thesis is being written, to
% \textit{name}. The \textit{name} is stored within the private
% |\thesis@departmentEn| macro.
%    \begin{macrocode}
\thesis@def{departmentEn}
\define@key{thesis}{departmentEn}{%
  \def\thesis@departmentEn{#1}}
%    \end{macrocode}
% \end{macro}
% \begin{macro}{\thesis@programme}
% \subsubsection{The \texttt{programme} key}
% The \marg{\texttt{programme}=name} pair sets the name of the
% author's study programme to \textit{name}. The \textit{name}
% is stored within the private |\thesis@programme| macro.
%    \begin{macrocode}
\thesis@def{programme}
\define@key{thesis}{programme}{%
  \def\thesis@programme{#1}}
%    \end{macrocode}
% \end{macro}
% \begin{macro}{\thesis@programmeEn}
% \subsubsection{The \texttt{programmeEn} key}
% The \marg{\texttt{programmeEn}=name} pair sets the English name
% of the author's study programme to \textit{name}. The
% \textit{name} is stored within the private |\thesis@programmeEn|
% macro.
%    \begin{macrocode}
\thesis@def{programmeEn}
\define@key{thesis}{programmeEn}{%
  \def\thesis@programmeEn{#1}}
%    \end{macrocode}
% \end{macro}
% \begin{macro}{\thesis@field}
% \subsubsection{The \texttt{field} key}
% The \marg{\texttt{field}=name} pair sets the name of the
% author's field of stufy to \textit{name}. The \textit{name}
% is stored within the private |\thesis@field| macro.
%    \begin{macrocode}
\thesis@def{field}
\define@key{thesis}{field}{%
  \def\thesis@field{#1}}
%    \end{macrocode}
% \end{macro}
% \begin{macro}{\thesis@fieldEn}
% \subsubsection{The \texttt{fieldEn} key}
% The \marg{\texttt{fieldEn}=name} pair sets the English name of
% the author's field of stufy to \textit{name}. The \textit{name}
% is stored within the private |\thesis@fieldEn| macro.
%    \begin{macrocode}
\thesis@def{fieldEn}
\define@key{thesis}{fieldEn}{%
  \def\thesis@fieldEn{#1}}
%    \end{macrocode}
% \end{macro}
% \begin{macro}{\thesis@universityLogo}
% \subsubsection{The \texttt{universityLogo} key}
% The \marg{\texttt{universityLogo}=filename} pair sets the
% filename of the logo file to be used to \textit{filename}. The
% \textit{filename} is stored within the private
% |\thesis@universityLogo| macro, whose implicit value is
% \texttt{base}. The logo file is loaded from the
% |\thesis@logopath|\discretionary{}{}{}|\thesis@logo| path.
%    \begin{macrocode}
\def\thesis@universityLogo{base}
\define@key{thesis}{universityLogo}{%
  \def\thesis@universityLogo{#1}}
%    \end{macrocode}
% \end{macro}
% \begin{macro}{\thesis@facultyLogo}
% \subsubsection{The \texttt{facultyLogo} key}
% The \marg{\texttt{facultyLogo}=filename} pair sets the filename
% of the logo file to be used to \textit{filename}. The
% \textit{filename} is stored within the private
% |\thesis@facultyLogo| macro, whose implicit value is
% |\thesis@faculty|. The logo file is loaded from the
% |\thesis@logopath\thesis@logo| path.
%    \begin{macrocode}
\def\thesis@facultyLogo{\thesis@faculty}
\define@key{thesis}{facultyLogo}{%
  \def\thesis@facultyLogo{#1}}
%    \end{macrocode}
% \end{macro}
% \begin{macro}{\thesis@style}
% \subsubsection{The \texttt{style} key}
% The \marg{\texttt{style}=filename} pair sets the filename of the
% style file to be used to \textit{filename}. The \textit{filename}
% is stored within the private |\thesis@style| macro, whose
% implicit value is |\thesis@university/fithesis3-\thesis@faculty|.
% The style file is loaded from the
% |\thesis@stylepath\thesis@style| path.
%    \begin{macrocode}
\def\thesis@style{\thesis@university/fithesis3-\thesis@faculty}
\define@key{thesis}{style}{%
  \def\thesis@style{#1}}
%    \end{macrocode}
% \end{macro}
% \begin{macro}{\thesis@style@inheritance}
% \subsubsection{The \texttt{styleInheritance} key}
% The \marg{\texttt{styleInheritance}=bool} pair either enables,
% if \textit{bool} is \texttt{true} or unspecified, or disables the
% inheritance for style files. The effects of the inheritance
% are documented within the subsection documenting the
% |\thesis@load| macro. The setting can be tested using the
% |\ifthesis@style@inheritance| \ldots
% |\else| \ldots |\fi| conditional. Inheritance is enabled for
% style files by default.
%    \begin{macrocode}
\newif\ifthesis@style@inheritance\thesis@style@inheritancetrue
\define@key{thesis}{styleInheritance}[true]{%
  \def\@true{true}%
  \def\@arg{#1}%
  \ifx\@true\@arg%
    \thesis@style@inheritancetrue%
  \else%
    \thesis@style@inheritancefalse%
  \fi}
%    \end{macrocode}
% \end{macro}
% \begin{macro}{\thesis@locale}
% \subsubsection{The \texttt{locale} key}
% The \marg{\texttt{locale}=filename} pair sets the filename of the
% locale file(s) to be used to \textit{filename}. The
% \textit{filename} is stored within the private |\thesis@locale|
% macro, whose implicit value is the main language of either the
% \textsf{babel} or the \textsf{polyglossia} package, or
% \texttt{english}, when undefined. If the inheritance is disabled
% for locale files, the locale file is loaded from the
% |\thesis@localepath\thesis@locale| path.
%    \begin{macrocode}
\def\thesis@locale{%
  % Babel detection
  \ifx\languagename\undefined%
  english\else\languagename\fi}
\define@key{thesis}{locale}{%
  \def\thesis@locale{#1}}
%    \end{macrocode}
% \end{macro}
% \begin{macro}{\ifthesis@english}
% The English locale is special. Several parts of the document will
% typically be typeset in both the current locale and English.
% However, if the current locale is English, this would result in
% duplicity. To avoid this, the |\ifthesis@english| \ldots |\else|
% \ldots |\fi| conditional is made available for testing, whether
% or not the current locale is English.
%    \begin{macrocode}
\def\ifthesis@english{
  \expandafter\def\expandafter\@english\expandafter{\string%
  \english}%
  \expandafter\expandafter\expandafter\def\expandafter%
  \expandafter\expandafter\@locale\expandafter\expandafter%
  \expandafter{\expandafter\string\csname\thesis@locale\endcsname}%
  \expandafter\csname\expandafter i\expandafter f\ifx\@locale%
  \@english%
    true%
  \else%
    false%
  \fi\endcsname}
%    \end{macrocode}
% \end{macro}
% \begin{macro}{\thesis@locale@inheritance}
% \subsubsection{The \texttt{localeInheritance} key}
% The \marg{\texttt{localeInheritance}=bool} pair either enables,
% if \textit{bool} is \texttt{true} or unspecified, or disables the
% inheritance. The effects of the inheritance are
% documented within the subsection documenting the |\thesis@load| 
% macro. The setting can be tested using the
% |\ifthesis@locale@inheritance| \ldots
% |\else| \ldots |\fi| conditional. Inheritance is enabled for locale
% files by default.
%    \begin{macrocode}
\newif\ifthesis@locale@inheritance\thesis@locale@inheritancetrue
\define@key{thesis}{localeInheritance}[true]{%
  \def\@true{true}%
  \def\@arg{#1}%
  \ifx\@true\@arg%
    \thesis@locale@inheritancetrue%
  \else%
    \thesis@locale@inheritancefalse%
  \fi}
%    \end{macrocode}
% \end{macro}
% \subsubsection{The \texttt{date} key}
% The \marg{\texttt{date}=date} pair sets the date of the thesis
% defence to \textit{date}, where \textit{date} is a string
% in the \texttt{YYYY/MM/DD} format, where \texttt{YYYY} stands
% for full year, \texttt{MM} stands for month and \texttt{DD}
% stands for day. The \textit{date} is parsed and stored using
% the \DescribeMacro{\thesis@parseDate}|\thesis@parseDate| private
% macro within the following private macros:
% \begin{itemize}
%   \item\DescribeMacro{\thesis@date}|\thesis@date| -- The whole
%     date
%   \item\DescribeMacro{\thesis@year}|\thesis@year| -- The year
%   \item\DescribeMacro{\thesis@month}|\thesis@month| -- The month
%   \item\DescribeMacro{\thesis@day}|\thesis@day| -- The day of
%     month
%   \item\DescribeMacro{\thesis@season}|\thesis@season| -- Expands
%     to either:
%     \begin{itemize}
%       \item\texttt{winter} if \texttt{MM} $<7$.
%       \item\texttt{summer} if \texttt{MM} $\geq7$.
%     \end{itemize}
%   \item\DescribeMacro{\thesis@academicYear}|\thesis@academicYear|
%     -- The academic year of the given semester:
%     \begin{itemize}
%       \item\texttt{YYYY/YYYY}$+1$ in case of a summer semester
%       \item\texttt{YYYY}$-1$\texttt{/YYYY} in case of a winter
%            semester
%     \end{itemize}
% \end{itemize}
% To set up the default values, the |\thesis@parseDate| macro is
% called with the fully expanded |\the\year/\the\month/\the\day|
% string.
%    \begin{macrocode}
\def\thesis@parseDate#1/#2/#3|{{
  % Basic info
  \gdef\thesis@date{#1/#2/#3}%
  \gdef\thesis@year{#1}%
  \gdef\thesis@month{#2}%
  \gdef\thesis@day{#3}%
  
  % Season and academic year
  \newcount\@year \expandafter\@year \thesis@year \relax%
  \newcount\@month\expandafter\@month\thesis@month\relax%
  \ifnum\@month<7%
    \gdef\thesis@season{winter}%
    \advance\@year-1\edef\@yearA{\the\@year}%
    \advance\@year 1\edef\@yearB{\the\@year}%
  \else%
    \gdef\thesis@season{summer}%
                    \edef\@yearA{\the\@year}%
    \advance\@year 1\edef\@yearB{\the\@year}%
  \fi%
  \global\edef\thesis@academicYear{\@yearA/\@yearB}}}

\edef\thesis@date{\the\year/\the\month/\the\day}%
\expandafter\thesis@parseDate\thesis@date|%

\define@key{thesis}{date}{{%
  \edef\@date{#1}%
  \expandafter\thesis@parseDate\@date|}}
%    \end{macrocode}
% \begin{macro}{\thesis@place}
% \subsubsection{The \texttt{place} key}
% The \marg{\texttt{place}=place} pair sets the location of the
% faculty, at which the thesis is being prepared, to \textit{place}.
% The \textit{place} is stored within the private |\thesis@place|
% macro, whose implicit value is \texttt{Brno}.
%    \begin{macrocode}
\def\thesis@place{Brno}
\define@key{thesis}{place}{%
  \def\thesis@place{#1}}
%    \end{macrocode}
% \end{macro}
% \begin{macro}{\thesis@title}
% \subsubsection{The \texttt{title} key}
% The \marg{\texttt{title}=title} pair sets the title of the
% thesis to \textit{title}. The \textit{title} is stored within the
% private |\thesis@title| macro.
%    \begin{macrocode}
\thesis@def{title}
\define@key{thesis}{title}{%
  \def\thesis@title{#1}}
%    \end{macrocode}
% \end{macro}
% \begin{macro}{\thesis@TeXtitle}
% \subsubsection{The \texttt{TeXtitle} key}
% The \marg{\texttt{TeXtitle}=title} pair sets the \TeX\ title of
% the thesis to \textit{title}. The \textit{title} is used, when
% typesetting the title, whereas |\thesis@title| is a plain text,
% which gets included in the PDF header of the
% resulting document as well as in the \BibTeX\ file containing
% the bibliographical entry for the thesis. The \textit{title}
% is stored within the private |\thesis@TeXtitle| macro, whose
% implicit value is |\thesis@title|.
%    \begin{macrocode}
\def\thesis@TeXtitle{\thesis@title}
\define@key{thesis}{TeXtitle}{%
  \def\thesis@TeXtitle{#1}}
%    \end{macrocode}
% \end{macro}
% \begin{macro}{\thesis@titleEn}
% \subsubsection{The \texttt{titleEn} key}
% The \marg{\texttt{titleEn}=title} pair sets the English title of
% the thesis to \textit{title}. The \textit{title} is stored within
% the private |\thesis@titleEn| macro.
%    \begin{macrocode}
\thesis@def{titleEn}
\define@key{thesis}{titleEn}{%
  \def\thesis@titleEn{#1}}
%    \end{macrocode}
% \end{macro}
% \begin{macro}{\thesis@TeXtitleEn}
% \subsubsection{The \texttt{TeXtitleEn} key}
% The \marg{\texttt{TeXtitleEn}=title} pair sets the English \TeX\ 
% title of the thesis to \textit{title}. The \textit{title} is
% used, when typesetting the title, whereas |\thesis@titleEn| is a
% plain text. The \textit{title} is stored within the private
% |\thesis@TeXtitleEn| macro, whose implicit value is
% |\thesis@titleEn|.
%    \begin{macrocode}
\def\thesis@TeXtitleEn{\thesis@titleEn}
\define@key{thesis}{TeXtitleEn}{%
  \def\thesis@TeXtitleEn{#1}}
%    \end{macrocode}
% \end{macro}
% \begin{macro}{\thesis@keywords}
% \subsubsection{The \texttt{keywords} key}
% The \marg{\texttt{keywords}=list} pair sets the keywords of the
% thesis to the comma-delimited \textit{list}. The \textit{list}
% is stored within the private |\thesis@keywords| macro.
%    \begin{macrocode}
\thesis@def{keywords}
\define@key{thesis}{keywords}{%
  \def\thesis@keywords{#1}}
%    \end{macrocode}
% \end{macro}
% \begin{macro}{\thesis@TeXkeywords}
% \subsubsection{The \texttt{TeXkeywords} key}
% The \marg{\texttt{TeXkeywords}=list} pair sets the \TeX\ keywords
% of the thesis to the comma-delimited \textit{list}. The
% \textit{list} is used, when typesetting the keywords, whereas
% |\thesis@keywords| is a plain text. The \textit{list} is stored
% within the private |\thesis@TeXkeywords| macro.
%    \begin{macrocode}
\def\thesis@TeXkeywords{\thesis@keywords}
\define@key{thesis}{TeXkeywords}{%
  \def\thesis@TeXkeywords{#1}}
%    \end{macrocode}
% \end{macro}
% \begin{macro}{\thesis@keywordsEn}
% \subsubsection{The \texttt{keywordsEn} key}
% The \marg{\texttt{keywordsEn}=list} pair sets the English
% keywords of the thesis to the comma-delimited \textit{list}. The
% \textit{list} is stored within the private |\thesis@keywordsEn|
% macro.
%    \begin{macrocode}
\thesis@def{keywordsEn}
\define@key{thesis}{keywordsEn}{%
  \def\thesis@keywordsEn{#1}}
%    \end{macrocode}
% \end{macro}
% \begin{macro}{\thesis@TeXkeywordsEn}
% \subsubsection{The \texttt{TeXkeywordsEn} key}
% The \marg{\texttt{TeXkeywordsEn}=list} pair sets the English
% \TeX\ keywords of the thesis to the comma-delimited \textit{list}.
% The \textit{list} is used, when typesetting the keywords, whereas
% |\thesis@keywordsEn| is a plain text. The \textit{list} is stored
% within the private |\thesis@TeXkeywordsEn| macro.
%    \begin{macrocode}
\def\thesis@TeXkeywordsEn{\thesis@keywordsEn}
\define@key{thesis}{TeXkeywordsEn}{%
  \def\thesis@TeXkeywordsEn{#1}}
%    \end{macrocode}
% \end{macro}
% \begin{macro}{\thesis@abstract}
% \subsubsection{The \texttt{abstract} key}
% The \marg{\texttt{abstract}=text} pair sets the abstract of the
% thesis to \textit{text}. The \textit{text} is stored within the
% private |\thesis@abstract| macro.
%    \begin{macrocode}
\thesis@def{abstract}
\long\def\KV@thesis@abstract#1{%
  \long\def\thesis@abstract{#1}}
%    \end{macrocode}
% \end{macro}
% \begin{macro}{\thesis@abstractEn}
% \subsubsection{The \texttt{abstractEn} key}
% The \marg{\texttt{abstractEn}=text} pair sets the English
% abstract of the thesis to \textit{text}. The \textit{text}
% is stored within the private |\thesis@abstractEn| macro.
%    \begin{macrocode}
\thesis@def{abstractEn}
\long\def\KV@thesis@abstractEn#1{%
  \long\def\thesis@abstractEn{#1}}
%    \end{macrocode}
% \end{macro}
% \begin{macro}{\thesis@advisor}
% \subsubsection{The \texttt{advisor} key}
% The \marg{\texttt{advisor}=name} pair sets the thesis advisor's
% full name to \textit{name}. The \textit{name} is stored within
% the private |\thesis@advisor| macro.
%    \begin{macrocode}
\thesis@def{advisor}
\define@key{thesis}{advisor}{\def\thesis@advisor{#1}}
%    \end{macrocode}
% \end{macro}
% \begin{macro}{\thesis@thanks}
% \subsubsection{The \texttt{thanks} key}
% The \marg{\texttt{thanks}=text} pair sets the acknowledgement
% text to \textit{text}. The \textit{text} is stored within
% the private |\thesis@thanks| macro.
%    \begin{macrocode}
\long\def\KV@thesis@thanks#1{%
  \long\def\thesis@thanks{#1}}
%    \end{macrocode}
% \end{macro}
% \begin{macro}{\thesis@assignmentFiles}
% \subsubsection{The \texttt{assignment} key}
% The \marg{\texttt{assignment}=list} pair sets the comma-separated
% list of paths to the pdf files containing the thesis assignment
% to \textit{list}. The \textit{list} is stored within the
% |\thesis@assignmentFiles| private macro.
%    \begin{macrocode}
\define@key{thesis}{assignment}{%
  \def\thesis@assignmentFiles{#1}}
%    \end{macrocode}
% \end{macro}
% \begin{macro}{\ifthesis@auto}
% \subsubsection{The \texttt{autoLayout} key}
% The \marg{\texttt{autoLayout}=bool} pair either enables,
% if \textit{bool} is \texttt{true} or unspecified, or disables
% autolayout. Autolayout injects the
% |\thesis@preamble| and |\thesis@postamble| private macros
% at the beginning and the end of the document, respectively. The
% setting can be tested using the |\ifthesis@auto| \ldots |\else|
% \ldots |\fi| conditional. The autolayout is enabled by default.
%    \begin{macrocode}
\newif\ifthesis@auto\thesis@autotrue
\define@key{thesis}{autoLayout}[true]{%
  \def\@true{true}%
  \def\@arg{#1}%
  \ifx\@true\@arg%
    \thesis@autotrue%
  \else%
    \thesis@autofalse%
  \fi}
%    \end{macrocode}
% \end{macro} ^^A The nested \ifthesis@auto macro definition
% \end{macro} ^^A The \thesissetup macro definition
% The \DescribeMacro{\thesis@preamble}|\thesis@postamble|
% and \DescribeMacro{\thesis@postamble}|\thesis@preamble|
% private macros are defined as empty strings by default and are
% subject to redefinition by the style files.
%    \begin{macrocode}
\def\thesis@preamble{}
\def\thesis@postamble{}
%    \end{macrocode}
% \subsection{The \cs{thesislong} macro}\label{sec:thesislong}
% \begin{macro}{\thesislong}
% The public macro |\thesislong|\marg{key}\marg{value},
% where \textit{value} may contain multiple paragraphs of text, can
% be used for the following \textit{key}s as an alternative to the
% |\thesissetup| public macro, which only permits a single
% paragraph as the \textit{value}:
% \begin{itemize}
%   \item\texttt{abstract}
%   \item\texttt{abstractEn}
%   \item\texttt{thanks}
%   \item\texttt{declaration}
% \end{itemize}
%    \begin{macrocode}
\long\def\thesislong#1#2{%
  \csname KV@thesis@#1\endcsname{#2}}
%    \end{macrocode}
% \end{macro}
% \section{Private API}
% \subsection{Main routine}\label{sec:thesis@load}
% \begin{macro}{\thesis@load}
% The |\thesis@load| macro is responsible for preparing the
% environment for, and consequently loading, the necessary locale
% and style files. By default, the |\thesis@load| macro gets
% expanded at the end of the preamble,
% but it can be inserted manually prior to that, if necessary to
% prevent package clashes. The \DescribeMacro{\ifthesis@loaded}
% |\ifthesis@loaded| semaphore ensures that the expansion is only
% performed once.
%    \begin{macrocode}
\newif\ifthesis@loaded\thesis@loadedfalse
\AtEndPreamble{\thesis@load}
\def\thesis@load{%
  \ifthesis@loaded\else%
    \thesis@loadedtrue
    \makeatletter%
%    \end{macrocode}
% First, the main locale file is loaded using the
% |\thesis@requireLocale| macro.
%    \begin{macrocode}
      \ifx\thesis@locale\empty\else
        \thesis@requireLocale{\thesis@locale}
      \fi
%    \end{macrocode}
% Consequently, the style files are loaded with the class options
% passed onto them.
%    \begin{macrocode}
      \ifx\thesis@style\empty\else
        \thesis@requireStyle{\thesis@style}
      \fi
%    \end{macrocode}
% With the placeholder strings loaded from the locale files, we
% can now inject metadata into the resulting PDF file. To this
% end, the \textsf{hyperref} package is conditionally included with
% the \texttt{unicode} option. Consequently, the following values
% are assigned to the PDF headers:\begin{itemize}
%   \item\texttt{Title} is set to |\thesis@title|.
%   \item\texttt{Author} is set to |\thesis@author|.
%   \item\texttt{Keywords} is set to |\thesis@keywords|.
%   \item\texttt{Creator} is set to \texttt{\thesis@version}.
% \end{itemize}
%    \begin{macrocode}
       \thesis@require{hyperref}%
      {\hypersetup{unicode,
         pdftitle={\thesis@title},%
         pdfauthor={\thesis@author},%
         pdfkeywords={\thesis@keywords},%
         pdfcreator={\thesis@version},%
     }}%
%    \end{macrocode}
% If autolayout is enabled, the |\thesis@preamble| and
% |\thesis@postamble| macros are scheduled for expansion at the
% beginning and at the end of the document, respectively.
%    \begin{macrocode}
      \ifthesis@auto%
        \AtBeginDocument{\thesis@preamble}%
        \AtEndDocument{\thesis@postamble}%
      \fi%
%    \end{macrocode}
% Lastly, a \BibTeX\ file named |\jobname.bib| containing the
% bibliographical entry for the thesis is scheduled to be
% generated at the end of the document in the working directory
% using the |\thesis@bibgen| macro and the
% \DescribeMacro{\thesis@pages}|\thesis@pages| private macro
% definition containing the length of the document is scheduled to
% be included in the auxiliary file.
%    \begin{macrocode}
      \AtEndDocument{%
        % Define \thesis@pages for the next run
        \write\@auxout{\noexpand\gdef\noexpand%
          \thesis@pages{\thepage}}}
    \makeatother%
  \fi}
%    \end{macrocode}
% \end{macro}
% \subsection{File manipulation macros}
% \begin{macro}{\thesis@exists}
% The |\thesis@exists|\marg{file}\marg{tokens} private macro is
% used to test for the existence of a given \textit{file}. If the
% \textit{file} exists, the macro expands to \textit{tokens}.
% Otherwise, a class warning is written to the output.
%    \begin{macrocode}
\def\thesis@input#1{%
  \thesis@exists{#1}{\input{#1}}}
%    \end{macrocode}
% \end{macro}\begin{macro}{\thesis@input}
% The |\thesis@input|\marg{file} private macro inputs the given
% \textit{file}, if it exists.
%    \begin{macrocode}
\def\thesis@exists#1#2{%
  \IfFileExists{#1}{#2}{%
  \ClassWarning{fithesis3}{File #1 doesn't exist}}}
%    \end{macrocode}
% \end{macro}\begin{macro}{\thesis@require}
% The |\thesis@require| \marg{package} expands to
% |\RequirePackage|\marg{package}, if the specified
% \textit{package} has not yet been loaded.
%    \begin{macrocode}
\def\thesis@require#1{%
  \@ifpackageloaded{#1}{}{\RequirePackage{#1}}}
%    \end{macrocode}
% \end{macro}\begin{macro}{\thesis@requireWithOptions}
% The |\thesis@requireWithOptions|\marg{package} expands to
% |\RequirePackageWithOptions|\marg{package}, if the specified
% \textit{package} exists and has not yet been loaded.
%    \begin{macrocode}
\def\thesis@requireWithOptions#1{\thesis@exists{#1.sty}{%
  \@ifpackageloaded{#1}{}{\RequirePackageWithOptions{#1}}}}
%    \end{macrocode}
% \end{macro}\begin{macro}{\thesis@requireStyle}
% If inheritance is enabled for style files, then the
% |\thesis@requireStyle|\marg{style} private macro sequentially
% loads each of the following files, provided they exist:
% \begin{enumerate}
%   \item|\thesis@stylepath fithesis3-base.sty|
%   \item|\thesis@stylepath\thesis@university/fithesis3-base.sty|
%   \item|\thesis@stylepath| \textit{style}|.sty|
% \end{enumerate}If inheritance is disabled for style files,
% then only the |\thesis@stylepath| \textit{style}|.sty| file is
% loaded. The \texttt{fithesis3-} prefix serves to prevent package
% clashes with other similarly named package files within the \TeX\
% directory structure.
%    \begin{macrocode}
\def\thesis@requireStyle#1{%
  \ifthesis@style@inheritance%
    \thesis@requireWithOptions{\thesis@stylepath fithesis3-base}%
    \thesis@requireWithOptions{\thesis@stylepath\thesis@university%
      /fithesis3-base}
  \fi%
  \thesis@requireWithOptions{\thesis@stylepath#1}}
%    \end{macrocode}
% \end{macro}\begin{macro}{\thesis@requireLocale}
% If inheritance is enabled for style files, then the
% |\thesis@requireStyle|\marg{locale} private macro sequentially
% loads each of the following locale files, provided they exist:
% \begin{enumerate}
%   \item|\thesis@localepath| \textit{locale}|.def|
%   \item|\thesis@localepath\thesis@university/|^^A
%     \textit{locale}|.def|
%   \item|\thesis@localepath\thesis@university/\thesis@faculty/|^^A
%     \textit{locale}|.def|
% \end{enumerate} If inheritance is disabled for locale files, then
% only the first listed file is used. The macro can be used within
% both locale and style files, although the usage within locale
% files is strongly discouraged to prevent circular dependencies.
%    \begin{macrocode}
\def\thesis@requireLocale#1{%
  % Prevent redundant entries
  \expandafter\ifx\csname thesis@#1@required\endcsname\relax%
    \expandafter\def\csname thesis@#1@required\endcsname{}%
      \thesis@input{\thesis@localepath#1.def}
      \ifthesis@locale@inheritance%
        \thesis@input{\thesis@localepath\thesis@university/#1.def}% 
        \thesis@input{\thesis@localepath\thesis@university/%
          \thesis@faculty/#1.def}% 
      \fi%
  \fi}
%    \end{macrocode}\end{macro}
% \subsection{String manipulation macros}
% \begin{macro}{\thesis@}
% The |\thesis@|\marg{name} macro expands to |\thesis@|
% \textit{name}, where \textit{name} gets fully expanded and can
% therefore contain active characters and command sequences.
%    \begin{macrocode}
\def\thesis@#1{\csname thesis@#1\endcsname}
%    \end{macrocode}
% \end{macro}\begin{macro}{\thesis@@}
% The |\thesis@@|\marg{name} macro expands to |\thesis@|
% \textit{locale}|@|\textit{name}, where \textit{locale}
% corresponds to the name of the current locale. 
% \textit{name} gets fully expanded and can
% therefore contain active characters and command sequences.
%    \begin{macrocode}
\def\thesis@@#1{\thesis@{\thesis@locale @#1}}
%    \end{macrocode}
% \end{macro}
% The \DescribeMacro{\thesis@lower}|\thesis@lower|
% and \DescribeMacro{\thesis@upper}|\thesis@upper|
% private macros are used for upper- and lowercasing within
% locale files. To cast the |\thesis@|\textit{name} macro
% to the lower- or uppercase, |\thesis@lower{|\textit{name}|}| or
% |\thesis@upper{|\textit{name}|}| would be used, respectively.
% \textit{name} gets fully expanded and can
% therefore contain active characters and command sequences.
%    \begin{macrocode}
\def\thesis@lower#1{{%
  \let\ea\expandafter%
  \ea\ea\ea\ea\ea\ea\ea\ea\ea\ea\ea\ea\ea\ea\ea\lowercase\ea\ea\ea
  \ea\ea\ea\ea\ea\ea\ea\ea\ea\ea\ea\ea{\ea\ea\ea\ea\ea\ea\ea\ea\ea
  \ea\ea\ea\ea\ea\ea\@gobble\ea\ea\ea\string\ea\csname\csname the%
  sis@#1\endcsname\endcsname}}}
\def\thesis@upper#1{{%
  \let\ea\expandafter%
  \ea\ea\ea\ea\ea\ea\ea\ea\ea\ea\ea\ea\ea\ea\ea\uppercase\ea\ea\ea
  \ea\ea\ea\ea\ea\ea\ea\ea\ea\ea\ea\ea{\ea\ea\ea\ea\ea\ea\ea\ea\ea
  \ea\ea\ea\ea\ea\ea\@gobble\ea\ea\ea\string\ea\csname\csname the%
  sis@#1\endcsname\endcsname}}}
%    \end{macrocode}
% The \DescribeMacro{\thesis@@lower}|\thesis@@lower|
% and \DescribeMacro{\thesis@@upper}|\thesis@@upper|
% private macros are used for upper- and lowercasing current
% \textit{locale} strings within style files. To cast the
% |\thesis@|\textit{locale}|@|\textit{name} macro to the
% lower- or uppercase, |\thesis@@lower{|\textit{name}|}| or
% |\thesis@@upper{|\textit{name}|}| would be used,
% respectively. \textit{name} gets fully expanded and can
% therefore contain active characters and command sequences.
%    \begin{macrocode}
\def\thesis@@lower#1{\thesis@lower{\thesis@locale @#1}}
\def\thesis@@upper#1{\thesis@upper{\thesis@locale @#1}}
%    \end{macrocode}
% The \DescribeMacro{\thesis@head}|\thesis@head|
% and \DescribeMacro{\thesis@tail}|\thesis@tail|
% private macros are used for retrieving a head or a tail of
% space-separated token sequences, which end with |\relax|.
%    \begin{macrocode}
\def\thesis@head#1 #2{%
  \ifx\relax#2%
    \expandafter\@gobbletwo%
  \else%
    \ #1%
  \fi%
  \thesis@head#2}%
\def\thesis@tail#1 #2{%
  \ifx\relax#2%
    #1%
    \expandafter\@gobbletwo%
  \fi%
  \thesis@tail#2}%
%    \end{macrocode}
% \subsection{General purpose macros}
% The \DescribeMacro{\thesis@pages}|\thesis@pages| macro is defined
% at the beginning of the second \LaTeX\ run as a part of the main
% routine (see Section \ref{sec:thesis@load}). During the first
% run, the macro expands to \texttt{??}.
%    \begin{macrocode}
\ifx\thesis@pages\undefined\def\thesis@pages{??}\fi
%    \end{macrocode}
% \iffalse
%</class>
% ^^A Old fithesis classes
%<*oldclass1>

\NeedsTeXFormat{LaTeX2e}
\ProvidesClass{oldfithesis1}[2015/03/04 old fithesis will load fithesis3 MU thesis class]

\ClassWarning{oldfithesis1}{%
  You are using the fithesis class, which has been deprecated.
  The fithesis3 class will be used instead.
  For more information, see <http://www.fi.muni.cz/tech/unix/tex/fithesis.xhtml>%
}\LoadClass{fithesis3}

%</oldclass1>
%
%<*oldclass2>

\NeedsTeXFormat{LaTeX2e}
\ProvidesClass{oldfithesis2}[2015/03/04 old fithesis2 will load fithesis3 MU thesis class]

\ClassWarning{oldfithesis2}{%
  You are using the fithesis2 class, which has been deprecated.
  The fithesis3 class will be used instead.
  For more information, see <http://www.fi.muni.cz/tech/unix/tex/fithesis.xhtml>%
}\LoadClass{fithesis3}

%</oldclass2>
% \fi
%
% \subsection{Locale files}
% \label{sec:locale-files}
% Locale files contain macro definitions for various locales. They
% live in the \texttt{locale/} subtree and they are loaded during
% the main routine (see Section \ref{sec:thesis@load}).
%
% When creating a new locale file, it is advisable to create one
% self-contained \texttt{dtx} file, which is then partitioned into
% locale files via the \textsf{docstrip} tool based on the
% respective \texttt{ins} file. A \DescribeMacro{\file} macro
% |\file|\marg{filename} is available for the sectioning the
% documentation of various files within the \texttt{dtx} file.
% \textit{filename}. For more information about \texttt{dtx} files
% and the \textsf{docstrip} tool, consult the \textsf{dtxtut,
% docstrip, doc} and \textsf{ltxdoc} manuals.
%
% \subsubsection{Interface}
% The union of locale files named \textit{locale}\texttt{.def},
% where \textit{locale} is the result of the expansion of
% |\thesis@locale|, loaded via main routine's inheritance scheme
% (see Section \ref{sec:thesis@load}) needs to define the following
% private macros:
% \begin{itemize}
%   \item|\thesis@|\textit{locale}|@universityName| -- The name of
%     the university
%   \item|\thesis@|\textit{locale}|@facultyName| -- The name of the
%     faculty
%   \item|\thesis@|\textit{locale}|@assignment| -- Instructions to
%     replace the current page with the official thesis assignment
%   \item|\thesis@|\textit{locale}|@declaration| -- The declaration
%     text
%   \item|\thesis@|\textit{locale}|@fieldTitle| -- The title of
%     the field of study entry
%   \item|\thesis@|\textit{locale}|@advisorTitle| -- The title of
%     the advisor
%   \item|\thesis@|\textit{locale}|@authorTitle| -- The title of
%     the author
%   \item|\thesis@|\textit{locale}|@abstractTitle| -- The title of
%     the abstract section
%   \item|\thesis@|\textit{locale}|@keywordsTitle| -- The title of
%     the keywords section
%   \item|\thesis@|\textit{locale}|@thanksTitle| -- The title of
%     the acknowledgement section
%   \item|\thesis@|\textit{locale}|@declarationTitle| -- The title
%     of the declaration section
%   \item|\thesis@|\textit{locale}|@idTitle| -- The title of the
%     thesis author's identifier field
%   \item|\thesis@|\textit{locale}|@winter| -- The name of the
%     winter semester
%   \item|\thesis@|\textit{locale}|@summer| -- The name of the
%     summer semester
%   \item|\thesis@|\textit{locale}|@semester| -- The full name of
%     the current semester
%   \item|\thesis@|\textit{locale}|@typeName| -- The name of the
%     thesis type
% \end{itemize}
%
% \def\file#1{\paragraph{The \texttt{#1} file}}
% \subsubsection{English locale files}
% \input{locale/english.dtx}
% \subsubsection{Czech locale files}
% \input{locale/czech.dtx}
% \subsubsection{Slovak locale files}
% \input{locale/slovak.dtx}
%
% \subsection{Style files}
% \label{sec:style-files}
% Style files define the structure and the look of the resulting
% document. They live in the \texttt{style/} subtree and they are
% loaded during the main routine (see Section
% \ref{sec:thesis@load}).
%
% When creating a new style file, it is advisable to create one
% self-contained \texttt{dtx} file, which can contain several
% files to be extracted via the \textsf{docstrip} tool based on the
% respective \texttt{ins} file. A \DescribeMacro{\file} macro
% |\file|\marg{filename} is available for sectioning the
% documentation of various files within the \texttt{dtx} file.
% For more information about \texttt{dtx} files and the
% \textsf{docstrip} tool, consult the \textsf{dtxtut, docstrip,
% doc} and \textsf{ltxdoc} manuals.
%
% \subsubsection{Interface}
% The union of style files loaded via main routine's inheritance
% scheme (see Section \ref{sec:thesis@load}) should define at least
% one of the following private macros:
% \begin{itemize}
%   \item\DescribeMacro{\thesis@preamble}
%                      |\thesis@preamble| -- If autolayout is
%                      enabled, then this macro is expanded at the
%                      very beginning of the document.
%   \item\DescribeMacro{\thesis@postamble}
%                      |\thesis@postamble| -- If autolayout is
%                      enabled, then this macro is expanded at the
%                      very end of the document.
% \end{itemize}
%
% \subsubsection{Base style files}
% \input{style/base.dtx}
% \input{style/mu/base.dtx}
% \subsubsection{The style files of the Faculty of Informatics}
% \input{style/mu/fi.dtx}
% \subsubsection{The style files of the Faculty of Science}
% \input{style/mu/sci.dtx}
% \subsubsection{The style files of the Faculty of Arts}
% \input{style/mu/phil.dtx}
% \subsubsection{The style files of the Faculty of Education}
% \input{style/mu/ped.dtx}
% \subsubsection{The style files of the Faculty of Social Studies}
% \input{style/mu/fss.dtx}
% \subsubsection{The style files of the Faculty of Law}
% \input{style/mu/law.dtx}
% \subsubsection{The style files of the Faculty of Economics and
%   Administration}
% \input{style/mu/econ.dtx}
% \subsubsection{The style files of the Faculty of Medicine}
% \input{style/mu/med.dtx}
% \subsubsection{The style files of the Faculty of Sports Studies}
% \input{style/mu/fsps.dtx}
