% My own macros
% Typesets a CSplain logo
\newcommand{\CSplain}{
  \ensuremath{\mathcal{C}\!}\raisebox{-0.5ex}{S}plain
}


% Packages
\usepackage[english]{babel}     % Babel
\usepackage[utf8]{inputenc}     % UTF-8
\usepackage{hologo}             % \XeLaTeX{} and friends
\usepackage[                    % Microtype
  protrusion                      % Protrusion support
]{microtype}
\usepackage{setspace}           % Paragraph spacing
\usepackage{blindtext}          % Blind text
\usepackage[toc,page]{appendix} % Appendices
\usepackage[
  labelfont=bf                  % Typeset table captions in bold font
]{caption}
\usepackage{tabularx}           % Elastic tables
\let\oldtabularx\tabularx         % Altering the definition to allow for colors
\let\endoldtabularx\endtabularx
\renewenvironment{tabularx}
  {\rowcolors{1}{facultyxlight}{facultylight}\oldtabularx}
  {\endoldtabularx}

% Fithesis options
\thesistitle{The Form of Theses\\ Written in \LaTeX{}}
\thesissubtitle{Bachelor's thesis}
\thesisstudent{Vít Novotný}
\thesiswoman{false}
\thesislang{en}
\thesisfaculty{fi}
\thesisyear{Spring 2015}
\thesisadvisor{RNDr. Michal Růžička}
\thesislogopath{fithesis3/loga}

% Color definitions
\definecolor{pantone122}{HTML}{FFD451}
\definecolor{pantone300}{HTML}{0067C6}
\definecolor{cured}{RGB}{210,45,64}

% Bibliography
\bibliographystyle{plainnat}
\usepackage[square,sort,comma,numbers]{natbib}

% Index
\usepackage{makeidx}
\makeindex

% Glossary
\usepackage[xindy,acronym,toc,sanitize=none]{glossaries}
\makeglossaries
\newacronym{fi}{FI}{the Faculty of Informatics of \acrlong{mu}}
\newacronym{sci}{FS}{the Faculty of Science of \acrlong{mu}}
\newacronym{fa}{FA}{the Faculty of Arts of \acrlong{mu}}
\newacronym{fedu}{FEdu}{the Faculty of Education of \acrlong{mu}}
\newacronym{fss}{FSS}{the Faculty of Social Studies of \acrlong{mu}}
\newacronym{fsps}{FSpS}{the Faculty of Sports Studies of \acrlong{mu}}
\newacronym{flaw}{FLaw}{the Faculty of Law of \acrlong{mu}}
\newacronym{fea}{FEA}{the Faculty of Economics \& Administration of \acrlong{mu}}
\newacronym{lf}{FM}{the Faculty of Medicine of \acrlong{mu}}

\newacronym{mu}{MU}{the Masaryk University in Brno}
\newacronym{fel}{CTU FEL}{the Faculty of Electrical Engineering of the Czech Technical University in Prague}
\newacronym{ctu}{CTU}{the Czech Technical University in Prague}
\newacronym{cuni}{CUNI}{the Charles University in Prague}
\newacronym{ascii}{ASCII}{American Standard Code for Information Interchange \cite{ascii}}

\newglossaryentry{pval}{
  name = {$p$-value},
  sort = {p-value},
  description = {The least significance level $p$ at which we can refuse the given zero \gls{hypothesis}},
}

\newdualentry{mime}{MIME-type}{Multipurpose Internet Mail Extensions Type}{
  One of the several ways to identify the type of content inside a file. As its name suggest, it was originally designed as an extension to the e-mail protocol \cite{rfc2045,rfc2046,rfc2047,rfc6838,rfc4289,rfc2049} that would allow the transfer of kinds of data other than \gls{ascii}, such as multimedia and binary files.
}

\newglossaryentry{hypothesis}{
  name = hypothesis,
  description = {With significance testing, we have two orthogonal hypotheses: the zero hypothesis $f(\ldots)=c$ and the alternative hypothesis $f(\ldots)\not=c$, where $f(\ldots)$ is a function of the statistics of the random variables under scrutiny and $c\in\mathbb{R}$. The hypotheses can then be tested at various significance levels. The higher the significance level, the higher the probability of refusing the zero hypothesis in favour of the alternative hypothesis, but the higher is also the risk of error}
}

\newglossaryentry{ctan}{
  first = {the Comprehensive \TeX{} Archive Network (CTAN)},
  name = {CTAN},
  description = {A website where \gls{tex}-related material and software can be found for download \cite{ctan}}
}

\newglossaryentry{latex}{
  name = \LaTeX{},
  description = {A macro package on top of \gls{tex}, which allows for the separation of the design and the contents of a document},
}

\newglossaryentry{docclass}{
  name = \LaTeX{} document class,
  plural = \LaTeX{} document classes,
  description = {A specification of the type of a \gls{latex} document the author wants to typeset}
}

\newglossaryentry{tex}{
  name = \TeX{},
  description = {A document markup language and a typesetting system used for the publication of complex scientific documents}
}

\newglossaryentry{csplain}{
  name = \CSplain,
  sort = CSplain,
  description = {A software package, which simplifies the task of typesetting czech and slovak documents in plain \TeX{} using multibyte encodings \cite{csplain}}
}

\newglossaryentry{cslatex}{
  name = \CSLaTeX,
  sort = CSLaTeX,
  description = {A software package, which simplifies the task of typesetting czech and slovak documents in \LaTeX{}. \CSLaTeX{} has been obsoleted by the babel package \cite{cslatexvsbabel}}
}

\newglossaryentry{opmac}{
  name = OPmac,
  description = {A lightweight macro package, which extends plain \TeX{} to include some basic functionality offered by \gls{latex} \cite{opmac}}
}

\newglossaryentry{makefile}{
  name = makefile,
  plural = makefiles,
  description = {A file, which specifies the files and commands necessary to create one or more target files. The makefile is read by the \texttt{make} utility, when the creation of one or more target files is requested}
}

\newdualentry{cmfonts}{CM fonts}{Computer Modern fonts}{A collection of typefaces, which was designed by Donald E. Knuth and which is used in \gls{tex}}

\newdualentry{ecfonts}{EC fonts}{European Computer fonts}{An extension of \gls{cmfonts}, which adds support for european languages using latin script. Due to their low typographic quality \cite{cslatexvsbabel}, EC fonts have been obsoleted by \gls{lmfonts}}

\newglossaryentry{csfonts}{
  name = \CS fonts,
  sort = CS fonts,
  description = {An extension of \gls{cmfonts}, which adds support for the typesetting of czech and slovak documents. Although preferrable over \gls{ecfonts}, \CS fonts have been obsoleted by the \gls{lmfonts} due to the low typographic quality of their Type 1 version \cite{cslatexvsbabel}}
}

\newdualentry{lmfonts}{LM fonts}{Latin Modern fonts}{An extension of \gls{cmfonts}, which adds support for european languages using latin script. Due to their high typographic quality \cite{cslatexvsbabel}, LM fonts have obsoleted both \gls{ecfonts} and \gls{csfonts}}

\newglossaryentry{texpackage}{
  name = macro package,
  plural = macro packages,
  description = {A set of \gls{tex} macros and commands, which can be loaded in the preamble of a \gls{tex} document to add or alter existing functionality}
}

\newglossaryentry{ltxpackage}{
  name = \LaTeX{} package,
  plural = \LaTeX{} packages,
  sort = LaTeX package,
  description = {A set of \gls{latex} macros and commands, which can be loaded in the preamble of a \gls{latex} document to add or alter existing functionality}
}

\newglossaryentry{musthave}{
  name = \textcolor{red}{[MUST HAVE]},
  sort = MUST HAVE,
  description = {A feature, whose implementation is critical}
}

\newglossaryentry{pending}{
  name = \textcolor{brown}{[PENDING]},
  sort = PENDING,
  description = {A feature, whose implementation is pending}
}

\newglossaryentry{rejected}{
  name = \textcolor{gray}{REJECTED},
  sort = REJECTED,
  description = {A feature, whose implementation has been rejected}
}

\newglossaryentry{implemented}{
  name = \textcolor{BlueViolet}{IMPLEMENTED},
  sort = IMPLEMENTED,
  description = {A feature, whose implementation has been brought to a successful end}
}

\newdualentry{ps}{PS}{PostScript}{A page-description language, which allows the creation of documents, whose appearance is independant on the underlying hardware and software. Unlike the \gls{gls-pdf}, PostScript is a turing-complete programming language, which makes it possible to generate procedural vector graphics with it. For more information, see \cite{psspec}}

\newdualentry{eps}{EPS}{Encapsulated PostScript}{A subset of the \gls{gls-ps} language, which imposes a set of formal restrictions meant to decrease the complexity of the resulting document. Encapsulated Postscript is primarily used as a vector graphics format. For more information, see \cite{epsspec}}

\newglossaryentry{texeng}{
  name = {\TeX{} engine},
  plural = {\TeX{} engines},
  sort = {TeX engine},
  description = {A program, which can process documents in (usually a superset of) the \gls{tex} language. The baseline \TeX{} engine is the \TeX{} typesetting program described in \cite{texbook}. Knuth's \TeX{}, along with the \hologo{eTeX} engine, is the building block of modern \TeX{} engines like \hologo{pdfTeX}, \Hologo{XeTeX}, \Hologo{ConTeXt} or \Hologo{LuaTeX}, which each come with extensions of their own}
}

\longnewglossaryentry{charenc}{
  name={character encoding},
  description={A character encoding specifies how characters are going to be represented on the low level. The first character encoding in use was \gls{ascii}, which was standardized in 1963 and which encodes lowercase and uppercase letters of english alphabet, digits, punctuation, a space and several teletype control codes. \gls{ascii} encodes each character as a 7bit string.\vskip\parskip\hskip\parindent As time went on, a plathora of 8-bit encodings, which remained backwards compatible with \gls{ascii}, but used the additional bit to support the encoding of various non-english alphabet characters, became increasingly popular. In the Central Europe, the encodings of choice were ISO 8859-2 \cite{isolatin2} and Windows-1250. Character encodings allowed for easy text processing, as each character was exactly one byte long, but also introducted additional complexity when dealing with documents, which contained characters from several non-english alphabets at once.\vskip\parskip\hskip\parindent Nowadays, the most commonly used encoding is UTF-8 \cite{rfc3629}, which can encode any character present in the Unicode character table \cite{unicode6}. This comes at the cost of producing variable-length characters, which introduces additional overhead to the text processing, although this overhead is generally regarded as trivial}
}

\newdualentry{pdf}{PDF}{Portable Document Format}{A page-description format tailored specifically to allow the creation of documents, whose appearance is independant on the underlying hardware and software. For more information, see \cite{isopdf}}

\newdualentry{dvi}{DVI}{Device independent file format}{the output of the \gls{tex} typesetting program, which describes the layout of the document and which is both hardware and software-independent. The format lacks any font or graphics embedding facilities and therefore needs to be transcoded into another format like \gls{gls-ps} prior to printing}

\newglossaryentry{shouldhave}{
  name = \textcolor{orange}{[SHOULD HAVE]},
  sort = SHOULD HAVE,
  description = {A feature, whose implementation is highly desirable}
}

\newglossaryentry{couldhave}{
  name = \textcolor{blue}{[COULD HAVE]},
  sort = COULD HAVE,
  description = {A feature, whose implementation is mostly optional}
}

\newglossaryentry{wannahave}{
  name = \textcolor{ForestGreen}{[NEAT TO HAVE]},
  sort = NEAT TO HAVE,
  description = {A feature, whose implementation is fully optional}
}

\newglossaryentry{todo}{
  name = \textcolor{red}{TODO},
  sort = TODO,
  description = {A part of the text, whose completion is pending}
}

\glossarystyle{altlistgroup}

% Penalties
\clubpenalty=10000  % Zero tolerance for widows and orphans

\begin{document}
  
  % Front matter
  \FrontMatter
  \ThesisTitlePage

  \begin{ThesisDeclaration}
    \DeclarationText
    \AdvisorName
  \end{ThesisDeclaration}

  \begin{ThesisThanks}
    \blindtext
  \end{ThesisThanks}

  \begin{ThesisAbstract}
    \blindtext
  \end{ThesisAbstract}

  \begin{ThesisKeyWords}
    \blindtext
  \end{ThesisKeyWords}

  \tableofcontents
  \listoftables
  \listoffigures

  % Main matter
  \MainMatter
  \chapter{Introduction}
    \blindtext

  \chapter{Existing \emph{Fithesis} Codebase}
    \blindtext
    
    \section{\emph{Fithesis} Document Class}
    \blindtext

    \section{\emph{Fithesis2} Document Class}
    \blindtext

    \section{\emph{Xslt2} Module}
    \blindtext

  \chapter{Research}
  \blindtext

    \section{User Survey}
    During the course of December 2014, an online questionnaire was distributed amongst the students of \gls{mu}. The table \ref{table:survey-faculty} illustrates the distribution of the respondents across the faculties of \gls{mu} according to the claims of the respondents. It should be noted that several respondents claimed to study at more than one faculty of \gls{mu}, which is the reason for the distortion of the percentage. The raw data obtained from the questionnaire in czech is available in the \texttt{survey.csv} file in the directory containing the thesis.

    \begin{table}
      \begin{tabularx}{\typearea}{Xcr}
        \textbf{On which faculty of \gls{mu} do you study?} & \textbf{\#} & \textbf{\%} \\
        \hline
        \textbf{Faculty of Informatics}                  & 82          & 92,1 \\
        \textbf{Faculty of Science}                      & 3           & 3,4  \\
        \textbf{Faculty of Education}                    & 2           & 2,2  \\
        \textbf{Faculty of Social Studies}               & 2           & 2,2  \\
        \textbf{Faculty of Law}                          & 1           & 1,1  \\
        \textbf{Faculty of Medicine}                     & 1           & 1,1  \\
        \textbf{Faculty of Arts}                         & 1           & 1,1  \\
        \textbf{Faculty of Economics \& Administration}  & 1           & 1,1  \\
        \textbf{Faculty of Sports Studies}               & 0           & 0,0  \\
        \hline
        \textbf{Total}                        & \textbf{89} & \textbf{100,0}
      \end{tabularx}
      \caption{The distribution of the questionnaire respondents across faculties}
      \label{table:survey-faculty}
    \end{table}

    The overwhelming majority of respondents claimed that the highest degree for which they are studying is that of a bachelor (see table \ref{table:survey-type}) and that they are planning to use \gls{tex} or \gls{latex} to write their theses (see table \ref{table:survey-sw}). Most of those who claimed to be planning to use \gls{tex} or \gls{latex} also claimed to know about the existence of the fithesis \gls{docclass} and claimed to be planning to use it as a template for their theses or dissertations (see table \ref{table:survey-tex}).

    \begin{table}
      \begin{tabularx}{\typearea}{Xcr}
        \textbf{Which academic degree are you currently pursuing?} & \textbf{\#} & \textbf{\%} \\
        \hline
        \textbf{Bachelor's degree}            & 70          & 78,7 \\
        \textbf{Master's degree}              & 17          & 19,1 \\
        \textbf{Doctorate}                    & 2           & 2,2  \\
        \hline
        \textbf{Total}                        & \textbf{89} & \textbf{100,0}
      \end{tabularx}
      \caption{The highest academic degrees currently pursued by the respondents of the questionnaire}
      \label{table:survey-type}
    \end{table}

    \begin{table}
      \begin{tabularx}{\typearea}{Xcr}
        \textbf{Which application do you use / are you planning to use to write your thesis or dissertation?} & \textbf{\#} & \textbf{\%} \\
        \hline
        \textbf{\TeX{} / \LaTeX{}}            & 65          & 73,0 \\
        \textbf{Microsoft Office}             & 16          & 18,0 \\
        \textbf{Apache OpenOffice, LibreOffice
                or another free office
                software suite}               & 4           &  4,5 \\
        \textbf{Other}                        & 4           &  4,5 \\
        \textbf{Google Documents}             & 0           &  0,0 \\
        \hline
        \textbf{Total}                        & \textbf{89} & \textbf{100,0}
      \end{tabularx}
      \caption{The software the respondents of the questionnaire are using or planning to use to write their theses or dissertations}
      \label{table:survey-sw}
    \end{table}

    \begin{table}
      \begin{tabularx}{\typearea}{Xcr}
        \textbf{Are you planning to use the \gls{mu} fithesis \LaTeX{} class?} & \textbf{\#} & \textbf{\%} \\
        \hline
        \textbf{Yes}                                         & 47          & 72,3 \\
        \textbf{Maybe, I didn't know it existed}             & 10          & 15,4 \\
        \textbf{No, I'm going to use another \gls{docclass}} & 5           &  7,7 \\
        \textbf{Other}                                       & 2           &  3,1 \\
        \textbf{No, I'm going to use plain \gls{tex}}        & 1           &  1,5 \\
        \hline
        \textbf{Total}                                       & \textbf{65} & \textbf{100,0}
      \end{tabularx}
      \caption{The attitude towards using the fithesis class amongst those respondents who claimed to be using or planning to use \gls{tex} or \gls{latex} to typeset their theses or dissertations}
      \label{table:survey-tex}
    \end{table}

    Some of the respondents who claimed to be using or planning to use \gls{tex} or \gls{latex} to typeset their theses or dissertations also provided feedback regarding the fithesis class. Unknown to the author at the time of the survey was the fact that fithesis2 had never been made publicly available, despite the fact that works on fithesis2 were brought to a successful end during 2009 \cite{Filipcik09}. As a result of that, much of the feedback was regarding issues long fixed in fithesis2, such as requests for the change of the default \gls{charenc} from ISO 8859-2 to UTF-8 \cite[section 4.1]{Filipcik09} or bug reports regarding a missing metafont logo file \cite{fithesis2@fbd7a25}. Among the feedback relevant to fithesis2 were calls for a more extensive user and technical documentation \gls{pending} and a suggestion that the class should support the typesetting of printed and electronic versions of theses as separate documents \gls{pending}.

    \section{Statistical analysis of existing theses}
    \blindtext

    \section{Comparison with Existing Solutions}
    As a part of the initial research, I also assessed several templates for the typesetting of theses in order to gather ideas for immediate and future improvements of the fithesis2 class. Initially, I'm going to focus on templates used at czech technical universities. I will then broaden the scope to include both the templates of foreign universities and generic thesis typesetting classes for \gls{latex} from \gls{ctan}.

      \subsection{CTUStyle\index{CTUStyle|(} and CUStyle\index{CUStyle|(}}
      The first of the reviewed thesis typesetting templates were Petr Olšák's CTUStyle \cite{ctustyle} and CUStyle \cite{custyle} \glspl{texpackage}, which are used at \gls{fel} and at \gls{cuni}, respectively. The sole dependencies of these templates are the \inx{\gls{csplain}} and \inx{\gls{opmac}} \glspl{texpackage}.

      The templates are closely tied with the visual styles of the universities, which is mainly achieved through color-coding. Aside from black-and-white text, the CTUstyle \gls{texpackage} typesets various typographic elements in the \clrpicker{pantone300} Pantone 300 color, which makes for a visually pleasing combination. The CUstyle \gls{texpackage} uses the combination of black, gray and \clrpicker{cured} bright maroon to a similar end. By typesetting various typographic elements in the \clrpicker{pantone122} Pantone 122 color, which is a part of the visual identity of \gls{fi} \cite{filogo}, and by replacing the logotype of the faculty with its colored version, which was created as a part of Matúš Kominka's bachelor thesis \cite{Kominka08}, the fithesis2 \gls{docclass} could be likewise revitalized. \implemented{v0.3.00, 01, 06} Since each faculty of \gls{mu} has a color version of its logo \cite{muvis}, this redesign could be implemented without the loss of support for other faculties. \implemented{The logos were downloaded as monochromatic and direct color \gls{eps} files from \cite{muvis} and then transformed into \gls{pdf} files using the \texttt{epstopdf} \cite{epstopdf} unix utility.}

      Unlike with fithesis2, documents typeset with the CTUStyle and CUStyle templates are double-sided by default. The horizontal margins of the CTUStyle (32\,mm \cite[line 249]{ctustyleCode}) and CUStyle (31\,mm \cite[line 229]{custyleCode}) templates are also much smaller than those of fithesis2, which measure 19\,mm \cite[lines 968\nobreakdash--976]{fithesis2Code} plus a binding correction of 1\,inch \cite{latexlayout} = 45\,mm on the left side and 210\,mm (A4 width) minus 45\,mm (left margin) minus 127\,mm (text width \cite[lines~989, 1017, 1045]{fithesis2Code}) = 38\,mm on the right side. The margins of CTUStyle and CUStyle, in conjunction with the chosen font family, allow for up to 100 characters per line, which is not only wearying to the eye of an inexperienced reader \todo{cite}, but also against the recommendations of \gls{fi} \cite[section 3.2.3]{bpdpfi}. Although not explicitly stated, both decisions are have likely been made in order to reduce the required storage space for the archival of physical prints of theses. An option to remove the binding correction and to relax horizontal margins for online publishing could be implemented into fithesis3. \rejected{Changes to the type area would affect line breaking}
      \index{CTUStyle|)}
      \index{CUStyle|)}

      \subsection{Felthesis}
      %Felthesis


      \subsection{The Charles University in Prague}
      Along with Petr Olšák's CUStyle, \gls{cuni} also provides an official \gls{latex} template for the typesetting of theses \cite{cunisablona}. Rather than defining a new \gls{docclass}, the archive contains a \gls{makefile} and a skeleton \gls{latex} document using the report \gls{docclass} for both czech and english theses. The students then modify said document to suit their requirements. The template uses \gls{cslatex} rather than the babel \gls{ltxpackage}, which means that \gls{csfonts} rather than the superior \gls{lmfonts} \cite{cslatexvsbabel} are used by default.

      % VUT -- http://latex.feec.vutbr.cz/cz/latex/download/sablona-pro-bp-dp-v2-4/
      % CTAN: muthesis, ... from Kumar07
      % ... more from https://tug.org/pracjourn/2007-4/
      % Add code as figures rather than bibliographic citations

    \section{Shortcomings of the \emph{Fithesis} Document Classes}
    \blindtext

    \section{Usability and Maintainability}
    \blindtext

  \chapter{Results}
    \blindtext

    \section{\emph{Fithesis2} Class Improvements}  
    \blindtext

      \subsection{\Hologo{XeLaTeX} Engine Support} 
      \blindtext

      \subsection{OpenType Font Support} 
      \blindtext

    \section{Thesis Writing Guidelines} 
    \blindtext

  \chapter{Conclusions} 
    \blindtext

  % Bibliography
  \newpage
  {\footnotesize
  \singlespacing
  \nocite{*}
  \bibliography{database}}

  % Glossary
  \newpage
  \printglossaries

  % Index
  \printindex

  \begin{appendices}

    \chapter{Example Theses} 
      \blindtext

    \chapter{User Guide} 
      \blindtext

    \chapter{Technical Documentation} 
      \blindtext

  \end{appendices}

\end{document}
