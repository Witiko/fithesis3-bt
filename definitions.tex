% The size of the type area
\def\typearea{127mm}

\newcommand{\CS}{%
  \,$\cal{CS}$%
}

% The CSplain logo
\newcommand{\CSplain}{%
  \CS plain%
}

% The CSLaTeX logo
\newcommand{\CSLaTeX}{%
  \CS\LaTeX{}%
}

% Index and print
\newcommand{\inx}[1]{%
  \index{#1}#1%
}

\usepackage{xparse} % A macro for creating an abbreviated glossary entry
% see: <http://en.wikibooks.org/wiki/LaTeX/Bibliography_Management>
\DeclareDocumentCommand{\newdualentry}{ O{} O{} m m m m } {
  \newglossaryentry{gls-#3}{name={#5},text={#5\glsadd{#3}},
    description={#6},#1
  }
  \newacronym[see={[Glossary:]{gls-#3}},#2]{#3}{#4}{#5\glsadd{gls-#3}}
}

% Color picker
\newcommand{\clrpicker}[1]{%
  % The 0.7em is font-specific
  % It roughly represents caps heigth - baseline
  \textcolor{#1}{\rule{0.7em}{0.7em}}%
}

\newcommand{\marker}[3]{%
  \def\empty{}%
  \def\arg{#3}%
  \textcolor{#2}{[\gls{#1}\ifx\arg\empty\else: #3\fi]}%
}


% A todo marker
\newcommand{\todo}[1]{%
  \marker{todo}{red}{#1}%
}

% A pending marker
\newcommand{\pending}[1]{%
  \marker{pending}{brown}{#1}%
}

% A rejected marker
\newcommand{\rejected}[1]{%
  \marker{rejected}{gray}{#1}%
}

% An implemented marker
\newcommand{\partimp}[1]{%
  \marker{partimp}{BlueViolet}{#1}%
}

% An implemented marker
\newcommand{\implemented}[1]{%
  \marker{implemented}{ForestGreen}{#1}%
}

% Control sequence printer
\DeclareRobustCommand{\cs}[1]{\texttt{\char`\\#1}}
