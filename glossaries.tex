% MUNI faculties
\newacronym{fi}{FI}{the Faculty of Informatics of \acrlong{mu}}
\newacronym{sci}{Sci}{the Faculty of Science of \acrlong{mu}}
\newacronym{fa}{FF}{the Faculty of Arts of \acrlong{mu}}
\newacronym{fedu}{Ped}{the Faculty of Education of \acrlong{mu}}
\newacronym{fss}{FSS}{the Faculty of Social Studies of
\acrlong{mu}}
\newacronym{fsps}{FSpS}{the Faculty of Sports Studies of
\acrlong{mu}}
\newacronym{flaw}{FLaw}{the Faculty of Law of \acrlong{mu}}
\newacronym{fea}{Econ}{the Faculty of Economics \& Administration
of \acrlong{mu}}
\newacronym{lf}{Med}{the Faculty of Medicine of \acrlong{mu}}

\newacronym{slu}{SU}{the Silesian University in Opava}
\newacronym{fpf}{FPF SU}{the Faculty of Philosophy and Science in
Opava of \acrlong{slu}}
\newacronym{opf}{OPF SU}{the School of Business Administration in
Karviná of \acrlong{slu}}
\newacronym{upol}{UP}{the Palacký University in Olomouc}
\newacronym{upolsci}{FS UP}{the Faculty of Science of \gls{upol}}
\newacronym{vsb}{VŠB-TU}{the Technical University of Ostrava}
\newacronym{tul}{TUL}{the Technical University in Liberec}
\newacronym{vut}{BUT}{the Brno University of Technology}
\newacronym{mu}{MU}{the Masaryk University in Brno}
\newacronym{mendelu}{MENDELU}{the Mendel University in Brno}
\newacronym{fel}{CTU FEL}{the Faculty of Electrical Engineering of
the Czech Technical University in Prague}
\newacronym{ctu}{CTU}{the Czech Technical University in Prague}
\newacronym{cuni}{CUNI}{the Charles University in Prague}
\newacronym{ascii}{ASCII}{American Standard Code for Information
Interchange \cite{ascii}}

\newglossaryentry{pval}{
  name = {$p$-value},
  sort = {p-value},
  description = {The least significance level $p$ at which we can
  refuse the given null \gls{hypothesis}},
}

\newdualentry{mime}{MIME type}{Multipurpose Internet Mail
Extensions Type}{
  One of the several ways to identify the type of content inside a
  file. As its name suggest, it was originally designed as an
  extension to the e-mail protocol
  \cite{rfc2045,rfc2046,rfc2047,rfc6838,rfc4289,rfc2049} that would
  allow the transfer of kinds of data other than \gls{ascii}, such
  as multimedia and binary files
}

\newglossaryentry{magnum}{
  name = {magic number},
  description = {A pattern of bytes located typically in the header
  of a file, which are used to determine the type of a file at UNIX
  systems}}

\newglossaryentry{textwidth}{
  name = {text width},
  description = {The part of the page surrounded by page margins
  into which text or graphics can be placed. The text width of this
  thesis is \the\textwidth{} $\approx$ 127\,mm}
}

\newdualentry{xml}{XML}{Extensible Markup Language}{
  A text-based markup language, which is primarily used for the
  exchange of structured textual data over the Internet}

\newglossaryentry{xmllang}{
  name = {\gls{xml} language},
  description = {A set of all \gls{xml} documents compliant with
    a given \gls{schema}}}

\newglossaryentry{schema}{
  name = {\gls{xml} schema},
  description = {A set of restrictions imposed on the structure of
    a \gls{xml} document. Documents compliant with a schema are
    said to be written in a \gls{xmllang} defined by the
    schema}}

\newglossaryentry{docbook}{
  name = {DocBook},
  description = {A \gls{xmllang} for writing documentation. The
    documentation can be published in a number of formats including
    web pages, \gls{pdf} documents and electronic books}}

\newglossaryentry{hypothesis}{
  name = hypothesis,
  plural = hypotheses,
  description = {With significance testing, we have two orthogonal
  hypotheses: the null hypothesis $\theta=c$ and the alternative
  hypothesis $\theta\not=c$, where $\theta$ is a function of given
  characteristics of the random variables under scrutiny and
  $c\in\mathbb{R}$. The hypotheses can then be tested at various
  significance levels. The higher the significance level, the
  higher the probability of refusing the null hypothesis in favour
  of the alternative hypothesis, but the higher is also the risk of
  error}}

\newdualentry{env}{environment}{\gls{latex} environment}{
  A pair of macros in the form of \cs{begin}\texttt{\{\textit{name}%
  \}} and \cs{end}\texttt{\{\textit{name}\}}, where
  \texttt{\textit{name}} is the name of the environment. They are
  used to insert macros before and after content that can not be
  reliably passed as an argument to a macro}

\newglossaryentry{ctan}{
  first = {the Comprehensive \TeX{} Archive Network (CTAN)},
  name = {CTAN},
  description = {A website, where \gls{tex}-related material and
  software can be found for download \cite{ctan}}}

\newdualentry{latex}{\LaTeX{}}{\LaTeXe{}}{A \gls{format}, which is
built around the idea of separation of design and contents of a
document}

\newdualentry[plural={\gls{latex} document
classes}][shortplural={classes},longplural={\gls{latex} document
classes}]{docclass}{class}{\gls{latex} document class}{A set of
\gls{latex} macros, which define the layout of the resulting
document}

\newglossaryentry{tex}{
  name = \TeX{},
  description = {A typesetting language and its interpreter, which
    serve to produce complex documents of high typographical
    quality. The language comprises primitive commands, which can
    be stored within macros}}

\newglossaryentry{format}{
  name = \TeX{} format,
  description = {A set of macros on top of the language constructs
  of \gls{tex}. The macro definitions are processed by the
  \hologo{iniTeX} utility, which dumps the state of \gls{tex}
  into a format file afterwards. The \gls{format} file is used to
  speed up future initializations of the given format}
}

\newglossaryentry{plain}{
  name = \hologo{plainTeX},
  sort = plainTeX,
  description = {A \gls{format} created by the author of \gls{tex},
  Prof.\ Donald Ervin Knuth. \Hologo{plainTeX} forms the basis of
  other \glspl{format}}}

\newglossaryentry{csplain}{
  name = \CSplain,
  sort = CSplain,
  description = {A software package, which simplifies the task of
  typesetting Czech and Slovak documents in \gls{plain} using
  multibyte \glspl{charenc} \cite{csplain}}
}

\newglossaryentry{cslatex}{
  name = \CSLaTeX,
  sort = CSLaTeX,
  description = {A set of configuration files, which simplifies the
  task of typesetting Czech and Slovak documents in \LaTeX{}.
  \CSLaTeX{} has been obsoleted in favour of the \textsf{babel}
  \gls{ltxpackage} \cite{cslatexvsbabel}}}

\newglossaryentry{opmac}{
  name = OPmac,
  description = {A lightweight \gls{format}, which extends
  \gls{plain} to include some basic functionality offered by
  \gls{latex} \cite{opmac}}
}

\newglossaryentry{makefile}{
  name = makefile,
  plural = makefiles,
  description = {A file, which specifies the files and commands
  necessary to create one or more target files. The makefile is
  read by the \texttt{make} utility, when the creation of one or
  more target files is requested}
}

\newdualentry{dtx}{DTX}{Documented \TeX{} file}{
  A \TeX{} document, whose comments form a separate \TeX{}
  document, which can be typeset \cite{dtxtut}}

\newdualentry{ins}{INS}{\textsf{DocStrip} installer file}{
  A \TeX{} document, which loads the \textsf{DocStrip}
  \gls{texpackage} and instructs it to decompose specified input
  files marked up with \textsf{DocStrip}-specific delimiter strings
  into specified output files. All comments are stripped in the
  output files \cite{docstrip}}

\newdualentry{cmfonts}{CM fonts}{Computer Modern fonts}{A
collection of typefaces, which was designed by Prof.\ Donald Ervin
Knuth and which is used in \gls{tex}}

\newdualentry{ecfonts}{EC fonts}{European Computer fonts}{An
extension of \gls{cmfonts}, which adds support for European
languages using Latin script. Due to their low typographic quality
\cite{cslatexvsbabel}, EC fonts have been obsoleted by
\gls{lmfonts}}

\newglossaryentry{csfonts}{
  name = \CS{} fonts,
  sort = CS fonts,
  description = {An extension of \gls{cmfonts}, which adds support
  for the typesetting of Czech and Slovak documents. Although
    preferable over \gls{ecfonts}, \CS{} fonts have been obsoleted
    by the \gls{lmfonts} due to the low typographic quality of
    their Type 1 version \cite{cslatexvsbabel}}
}

\newdualentry{lmfonts}{LM fonts}{Latin Modern fonts}{An extension
of \gls{cmfonts}, which adds support for European languages using
Latin script. Due to their high typographic quality
\cite{cslatexvsbabel}, LM fonts have obsoleted both \gls{ecfonts}
and \gls{csfonts}}

\newglossaryentry{texpackage}{
  name = macro package,
  plural = macro packages,
  description = {A set of \gls{tex} macros and commands, which can
  be included in the preamble of a \gls{tex} document to add new or
  alter existing functionality}}

\newdualentry[plural={\gls{latex}
packages}][shortplural={packages},longplural={\gls{latex}
packages}]{ltxpackage}{package}{\gls{latex} package}{A set of
\gls{latex} macros and commands, which can be loaded in the
preamble of a \gls{latex} document to add new or alter existing
functionality}

\newglossaryentry{biblatex}{
  name = \BibLaTeX{},
  sort = BibLaTeX,
  description = {A \gls{ltxpackage} for the automated typesetting
  of bibliography stored in a separate database file}
}

\newglossaryentry{beamer}{
  name = beamer,
  description = {A \acrlong{docclass} for the typesetting of
  presentations}  
}

\newglossaryentry{pending}{
  sort = PENDING,
  name = \textcolor{pending}{PENDING},
  description = {A feature, whose implementation is pending}
}

\newglossaryentry{remark}{
  sort = REMARK,
  name = \textcolor{Blue}{REMARK},
  description = {A suggestion by a reader of this thesis}
}
\newglossaryentry{partimp}{
  sort = PARTIALLY IMPLEMENTED,
  name = \textcolor{BlueViolet}{PARTIALLY IMPLEMENTED},
  description = {A feature, whose implementation has been partially
  completed}
}

\newglossaryentry{implemented}{
  sort = IMPLEMENTED,
  name = \textcolor{ForestGreen}{IMPLEMENTED},
  description = {A feature, whose implementation has been brought
  to a successful end}
}

\newglossaryentry{zip}{
  name = ZIP,
  description = {An archive file format that allows the user to
  include the contents of a directory tree into a single
  file. The contents can be optionally compressed using one of
  the several supported compression algorithms}}

\newdualentry{csv}{CSV}{comma-separated values}{A plain text file
  format for the storage of tabular data. Individual cells are
  separated by commas and rows are separated by line breaks
  \cite{rfc4180}}

\newdualentry{ps}{PS}{PostScript}{A page-description language,
which allows the creation of documents, whose appearance is
independent on the underlying hardware and software \cite{psspec}.
Unlike the \gls{gls-pdf}, Postscript is a Turing-complete
programming language, which enables procedural vector graphics
generation.}

\newdualentry{eps}{EPS}{Encapsulated \gls{gls-ps}}{A subset of the
\gls{gls-ps} language, which imposes a set of formal restrictions
with the intent to decrease the complexity of the resulting
document \cite{epsspec}. Encapsulated \gls{gls-ps} is primarily
used as a vector graphics format}

\newdualentry{tds}{TDS}{\gls{tex} Directory Structure}{A set of
rules and recommendations \cite{tds} describing a unified directory
structure containing \gls{tex} distribution files such as fonts,
\glspl{format}, \glspl{ltxpackage} and \glspl{docclass}. The
\gls{tex} directory structure is used by all major \gls{tex}
distributions}

\newdualentry{texeng}{engine}{\gls{tex} engine}{An interpreter of
(usually a superset of) the \gls{tex} language. The baseline
\gls{tex} engine, whose language extensions are supported by
virtually all modern \TeX{} engines like \hologo{pdfTeX},
\Hologo{XeTeX} or \Hologo{LuaTeX}, is \hologo{eTeX}}

\longnewglossaryentry{charenc}{
  name={character encoding},
  description={Specifies how characters are going to be represented
  on the bit level. The first character encoding in use was
  \gls{ascii}, which was standardized in 1963 and which encodes
  lowercase and uppercase letters of English alphabet, digits,
  punctuation, a space and several teletype control codes.
  \gls{ascii} encodes each character as a 7bit
  string.\vskip\parskip\hskip\parindent As time went on, a plethora
  of 8-bit encodings, which remained backwards compatible with
  \gls{ascii}, but used the additional bit to support the encoding
  of various non-English alphabet characters, became increasingly
  popular. In the Central Europe, the encodings of choice were ISO
  8859-2 \cite{isolatin2} and Windows-1250. Character encodings
  enabled easy text processing, as each character was exactly
  one byte long, but also introduced additional complexity when
  dealing with documents, which contained characters from several
  non-English alphabets at once.\vskip\parskip\hskip\parindent
  Nowadays, the most commonly used encoding is UTF-8
  \cite{rfc3629}, which can encode any character present in the
  Unicode character table \cite{unicode6}. This comes at the cost
  of producing variable-length characters, which introduces
  additional overhead to the text processing, although this
  overhead is generally regarded as trivial}}

\newdualentry{pdf}{PDF}{Portable Document Format}{A
page-description format tailored specifically to allow the creation
of documents, whose appearance is independent on the underlying
hardware and software \cite{isopdf}}

\newdualentry{dvi}{DVI}{Device independent file format}{The output
of the \gls{tex} typesetting program, which describes the layout of
the document and which is both hardware and software-independent.
The format lacks any font or graphics embedding facilities and
therefore needs to be transcoded into another format like
\gls{gls-ps} prior to printing}

\newglossaryentry{todo}{
  name = \textcolor{red}{TODO},
  sort = TODO,
  description = {A part of the text, whose completion is pending}
}

\newglossaryentry{metafont}{
  name = \MF{},
  sort = Metafont,
  description = {A language developed by Prof.\ Donald Ervin Knuth
  alongside \gls{tex}, which allows the definition of vector fonts}
}
