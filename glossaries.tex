\newacronym{fi}{FI MU}{the Faculty of Informatics of the Masaryk University in Brno}
\newacronym{sci}{FS MU}{the Faculty of Science of the Masaryk University in Brno}
\newacronym{mu}{MU}{the Masaryk University in Brno}
\newacronym{fel}{CTU FEL}{the Faculty of Electrical Engineering of the Czech Technical University in Prague}
\newacronym{ctu}{CTU}{the Czech Technical University in Prague}
\newacronym{cuni}{CUNI}{the Charles University in Prague}
\newacronym{ascii}{ASCII}{American Standard Code for Information Interchange \cite{ascii}}

\newglossaryentry{ctan}{
  first = {the Comprehensive \TeX{} Archive Network (CTAN)},
  name = {CTAN},
  description = {A website where \gls{tex}-related material and software can be found for download \cite{ctan}}
}

\newglossaryentry{latex}{
  name = \LaTeX{},
  description = {A macro package on top of \gls{tex}, which allows for the separation of the design and the contents of a document},
}

\newglossaryentry{docclass}{
  name = \LaTeX{} document class,
  plural = \LaTeX{} document classes,
  description = {A specification of the type of a \gls{latex} document the author wants to typeset}
}

\newglossaryentry{tex}{
  name = \TeX{},
  description = {A document markup language and a typesetting system used for the publication of complex scientific documents}
}

\newglossaryentry{csplain}{
  name = \CSplain,
  sort = CSplain,
  description = {A software package, which simplifies the task of typesetting czech and slovak documents in plain \TeX{} using multibyte encodings \cite{csplain}}
}

\newglossaryentry{cslatex}{
  name = \CSLaTeX,
  sort = CSLaTeX,
  description = {A software package, which simplifies the task of typesetting czech and slovak documents in \LaTeX{}. \CSLaTeX{} has been obsoleted by the babel package \cite{cslatexvsbabel}}
}

\newglossaryentry{opmac}{
  name = OPmac,
  description = {A lightweight macro package, which extends plain \TeX{} to include some basic functionality offered by \gls{latex} \cite{opmac}}
}

\newglossaryentry{makefile}{
  name = makefile,
  plural = makefiles,
  description = {A file, which specifies the files and commands necessary to create one or more target files. The makefile is read by the \texttt{make} utility, when the creation of one or more target files is requested}
}

\newdualentry{cmfonts}{CM fonts}{Computer Modern fonts}{A collection of typefaces, which was designed by Donald E. Knuth and which is used in \gls{tex}}

\newdualentry{ecfonts}{EC fonts}{European Computer fonts}{An extension of \gls{cmfonts}, which adds support for european languages using latin script. Due to their low typographic quality \cite{cslatexvsbabel}, EC fonts have been obsoleted by \gls{lmfonts}}

\newglossaryentry{csfonts}{
  name = \CS fonts,
  sort = CS fonts,
  description = {An extension of \gls{cmfonts}, which adds support for the typesetting of czech and slovak documents. Although preferrable over \gls{ecfonts}, \CS fonts have been obsoleted by the \gls{lmfonts} due to the low typographic quality of their Type 1 version \cite{cslatexvsbabel}}
}

\newdualentry{lmfonts}{LM fonts}{Latin Modern fonts}{An extension of \gls{cmfonts}, which adds support for european languages using latin script. Due to their high typographic quality \cite{cslatexvsbabel}, LM fonts have obsoleted both \gls{ecfonts} and \gls{csfonts}}

\newglossaryentry{texpackage}{
  name = macro package,
  plural = macro packages,
  description = {A set of \gls{tex} macros and commands, which can be loaded in the preamble of a \gls{tex} document to add or alter existing functionality}
}

\newglossaryentry{ltxpackage}{
  name = \LaTeX{} package,
  plural = \LaTeX{} packages,
  sort = LaTeX package,
  description = {A set of \gls{latex} macros and commands, which can be loaded in the preamble of a \gls{latex} document to add or alter existing functionality}
}

\newglossaryentry{musthave}{
  name = \textcolor{red}{[MUST HAVE]},
  sort = MUST HAVE,
  description = {A feature, whose implementation is critical}
}

\newglossaryentry{pending}{
  name = \textcolor{brown}{[PENDING]},
  sort = PENDING,
  description = {A feature, whose implementation is pending}
}

\newglossaryentry{rejected}{
  name = \textcolor{gray}{REJECTED},
  sort = REJECTED,
  description = {A feature, whose implementation has been rejected}
}

\newglossaryentry{implemented}{
  name = \textcolor{BlueViolet}{IMPLEMENTED},
  sort = IMPLEMENTED,
  description = {A feature, whose implementation has been brought to a successful end}
}

\newglossaryentry{eps}{
  first = Encapsulated PostScript,
  name = EPS,
  description = {A document in the PostScript language, which is not only well-formed, but also satisfies a set of formal restrictions meant to restrict the complexity of the resulting document. Used primarily as a vector graphics format. For more information, see \cite{epsspec}}
}

\longnewglossaryentry{charenc}{
  name={character encoding},
  description={A character encoding specifies how characters are going to be represented on the low level. The first character encoding in use was \gls{ascii}, which was standardized in 1963 and which encodes lowercase and uppercase letters of english alphabet, digits, punctuation, a space and several teletype control codes. \gls{ascii} encodes each character as a 7bit string.\vskip\parskip\hskip\parindent As time went on, a plathora of 8-bit encodings, which remained backwards compatible with \gls{ascii}, but used the additional bit to support the encoding of various non-english alphabet characters, became increasingly popular. In the Central Europe, the encodings of choice were ISO 8859-2 \cite{isolatin2} and Windows-1250. Character encodings allowed for easy text processing, as each character was exactly one byte long, but also introducted additional complexity when dealing with documents, which contained characters from several non-english alphabets at once.\vskip\parskip\hskip\parindent Nowadays, the most commonly used encoding is UTF-8 \cite{rfc3629}, which can encode any character present in the Unicode character table \cite{unicode6}. This comes at the cost of producing variable-length characters, which introduces additional overhead to the text processing, although this overhead is generally regarded as trivial}
}

\newglossaryentry{pdf}{
  first = Portable Document Format,
  name = PDF,
  description = {A file format tailored specifically to allow the creation of documents, whose appearance is independant on the underlying hardware and software. For more information, see \cite{isopdf}}
}

\newglossaryentry{shouldhave}{
  name = \textcolor{orange}{[SHOULD HAVE]},
  sort = SHOULD HAVE,
  description = {A feature, whose implementation is highly desirable}
}

\newglossaryentry{couldhave}{
  name = \textcolor{blue}{[COULD HAVE]},
  sort = COULD HAVE,
  description = {A feature, whose implementation is mostly optional}
}

\newglossaryentry{wannahave}{
  name = \textcolor{ForestGreen}{[NEAT TO HAVE]},
  sort = NEAT TO HAVE,
  description = {A feature, whose implementation is fully optional}
}

\newglossaryentry{todo}{
  name = \textcolor{red}{TODO},
  sort = TODO,
  description = {A part of the text, whose completion is pending}
}
